\chapter{Linear Algebra Basics}

\section{Linear Functions}

A function $L \colon \R^m \rightarrow \R^n$ is a linear function if for all
$\vect{x}, \vect{y} \in \R^m$ and for all $a, b \in \R$ 
\[
    L(a \vect{x} + b \vect{y}) = a L(\vect{x}) + b L(\vect{y}).
\]
It follows (by induction) that for all $\vect{x}_1, \ldots, \vect{x}_r \in \R^m$
and all $a_1, \ldots, a_r \in \R$
\[
    L(a_1 \vect{x}_1 + \cdots + a_r \vect{x}_r) = 
        a_1 L(\vect{x}_1) + \cdots + a_r L(\vect{x}_r).
\]

\begin{theorem}
A linear function $L \colon \R^m \rightarrow \R^n$ is completely determined by 
its effect on the standard basis vectors $\vect{e}_1, \ldots, \vect{e}_m$ of $\R^m$.
An arbitrary choice of vectors $L(\vect{e}_1), \ldots, L(\vect{e}_m)$ of $\R^n$ 
determines a linear function from $\R^m$ to $\R^n$.
\end{theorem}
\begin{proof}
Given any vector $\vect{x} \in \R^m$, we can express it as a unique linear
combination~$ \sum_{i = 1}^m \alpha_i \vect{e}_i$ of the basis vectors.  By the
linearity of $L$, $L(\vect{x}) = \sum_i \alpha_i L(\vect{e}_i)$ which 
is completely specified by $L(\vect{e}_1), \ldots, L(\vect{e}_m)$.

Let $\vect{b}_1, \ldots, \vect{b}_m$ be any vectors in $\R^n$. Define a 
map~$L$ from $\R^m$ to $\R^n$ as follows: for 
$\vect{x} = \sum_{i = 1}^m \alpha_i \vect{e}_i \in \R^m$, 
$L(\vect{x}) = \sum_{i = 1}^m \alpha_i \vect{b}_i$. Then 
$L(\vect{e}_i) = \vect{b}_i$ for all $1 \leq i \leq m$ and for all
$\vect{x}, \vect{y} \in \R^m$ and all $a, b \in \R$:
\[
    L(a \vect{x} + b \vect{y}) = \sum_{i = 1}^m (a x_i + b y_i) \vect{b}_i 
        = a \sum_{i} x_i \vect{b}_i + b \sum_{i} y_i \vect{b}_i 
        = a L(\vect{x}) + b L(\vect{y})
\] 
\end{proof}

Note that the domain of definition of a linear function must be a vector space. An 
non-linear function can be defined on a subset of a vector space.

\section{Image and Kernel of a Linear Function}
The image~$\image (L)$ of a linear function~$L \colon \R^m \rightarrow \R^n$ is
the set of vectors in~$\R^n$ that $L$ maps~$\R^m$ to. In symbols, $\image (L)
:= \{ L (\vect{x}) \in \R^n \colon \vect{x} \in \R^m\}$.  The kernel~$\kernel
(L)$ of~$L$ is the set of vectors in~$\R^m$ that $L$ maps to the zero vector
in~$\R^n$: $\kernel (L) := \{\vect{x} \in \R^m \colon L(\vect{x}) =
\vect{0}_n\}$.

\begin{theorem}
Let $L \colon \R^m \rightarrow \R^n$ be a linear function. Then the following
hold:
\begin{enumerate}
	\item $\image (L)$ is a subspace of $\R^n$;
	\item $\kernel (L)$ is a subspace of $\R^m$;
	\item $\dim (\image (L)) + \dim (\kernel (L)) = \dim (\R^m) = m$.
\end{enumerate}
\end{theorem}
\begin{proof}
The proof of~$(1)$ and~$(2)$ are similar. For proving~$(1)$, let 
$\vect{z}_1, \vect{z}_2 \in \image(L)$ and $\alpha, \beta \in \R$. 
Then there exist $\vect{x}_1, \vect{x}_2 \in \R^m$ such that 
\[
	L(\alpha \vect{x}_1 + \beta \vect{x}_2) = \alpha  \vect{z}_1 + \beta \vect{z}_2,
\]
implying that $\alpha  \vect{z}_1 + \beta \vect{z}_2 \in \image (L)$. Since 
$\image(L)$ is closed under vector addition and multiplication by scalars, it is 
a vector space. Since it is a subset of $\R^n$, it must be a subspace of $\R^n$.

To prove~$(3)$, let us assume that $\dim ( \kernel (L)) = k$ and that 
$\vect{b}_1, \ldots, \vect{b}_k$ is a basis of $\kernel (L)$. Extend this 
basis to a basis $\vect{b}_1, \ldots, \vect{b}_k, \vect{b}_{k + 1}, \ldots, \vect{b}_m$
for $\R^m$. Then for every vector $\vect{x} \in \R^m$ there exist scalars 
$\beta_1, \ldots, \beta_m$ such that $\vect{x} = \sum_{i = 1}^k \beta_i \vect{b}_i$. 
Moreover since $L$ is linear,
\begin{align*}
	L(\vect{x}) & = \sum_{i = 1}^{m} \beta_i L(\vect{b}_i) \\
				& = \sum_{i = k + 1}^{m} \beta_i L(\vect{b}_i).
\end{align*}
This shows that every vector in $\image (L)$ can be expressed as a linear 
combination of the vectors $L(\vect{b}_{k + 1}), \ldots, L(\vect{b}_m)$. 
To show that they form a basis of $\image (L)$, it is sufficient to show
that they are linearly independent. Suppose not. Then there exist scalars 
$\beta_{k + 1}, \ldots, \beta_m$, not all zero, such that 
$\sum_{i = k + 1}^m \beta_i L(\vect{b}_i) = \vect{0}_n$. By the linearity of~$L$, 
we have $L(\sum_{i = k + 1}^m \beta_i \vect{b}_i) = \vect{0}_n$ and hence 
$\vect{0}_m \neq \sum_{i = k + 1}^m \beta_i \vect{b}_i \in \kernel (L)$. 
This is a contradiction since the vectors $\vect{b}_{k + 1}, \ldots, \vect{b}_m$
are not in the space spanned by the vectors $\vect{b}_1, \ldots, \vect{b}_k$. 
Thus the vectors $\beta_{k + 1}, \ldots, \beta_m$ must be independent and form
a basis of $\image (L)$. This proves~$(3)$. 
\end{proof}

\section{Quadratic Forms}

A function $q \colon \R^m \rightarrow \R$ is called a \emph{quadratic form} 
if there exists a real symmetric $m \times m$ matrix $A$ such that for all 
$\vect{x} \in \R^m$ such that
\[
	q(\vect{x}) = \trans{\vect{x}} A \vect{x}.
\]
The right-hand side of the equation may be written as:
\begin{align*}
(x_1, \ldots, x_m) \begin{pmatrix} 
						a_{11} & a_{12} & \cdots & a_{m 1} \\
						\vdots & \vdots & \cdots & \vdots \\
						a_{m1} & a_{m2} & \cdots & a_{mm} 	
                   \end{pmatrix} 
				   \begin{pmatrix}
						x_1 \\
						\vdots \\
						x_m
				   \end{pmatrix} & = (x_1, \ldots, x_m) [x_1 A_{* 1} + \cdots + x_m A_{* m}] \\
								 & = \sum_{i = 1}^m \sum_{j = 1}^m x_i a_{ij} x_j.
\end{align*} 
The symbol $A_{* j}$ denotes the $j$th column of $A$. The name ``quadratic'' form arises 
from the last expression which is what a quadratic expression in $m$ variables looks like.
\begin{example}
The function 
\[
	q(h_1, h_2, h_3) = h_1^2 + h_2^2 + h_3^2 + 2h_1h_2 + 2h_2h_3 + 2h_1h_3 = (h_1 + h_2 + h_3)^2
\]
can be expressed as $\trans{\vect{h}} A \vect{h}$, where $A$ is the all-ones $3 \times 3$ matrix. 
\end{example}

\subsection{Definiteness}
A quadratic form $q \colon \R^m \rightarrow \R$ is 
\begin{enumerate}
	\item \emph{positive definite} if for all non-zero $\vect{x} \in \R^m$, $q(\vect{x}) > 0$;
	\item \emph{negative definite} if for all non-zero $\vect{x} \in \R^m$, $q(\vect{x}) > 0$;
	\item \emph{indefinite} if $q(\vect{x})$ takes on both positive and negative values.  
\end{enumerate}

\begin{example}
Let $q(\vect{h}) = -h_1^2 + 2h_2^2 - h_3^2$ such that the matrix associated with~$q$ is
\[
	\begin{pmatrix}
	-1 & 0 & 0 \\
	0  & 2 & 0 \\
    0  & 0 & -1 
	\end{pmatrix}.
\]
The $q(0, 1, 0) = 2$ and $q(1, 0, 1) = -2$. Thus the quardatic form $q$ is indefinite.
\end{example}    

\begin{example}
The quadratic form $q(h_1, h_2, h_3) = h_1^2 + h_2^2 + 4h_3^2 + 2h_1h_2 - 4h_2h_3 - 4h_1 h_3$
is positive definite, since we may re-write $q(h_1, h_2, h_3)$ as $(h_1 + h_2 - 2h_3)^2$
which is strictly positive for all $(h_1, h_2, h_3) \in \R^3 \setminus \{\vect{0}\}$.
\end{example}
To characterize the definiteness of quadratic forms, we make use of the following 
properties of real symmetric matrices.
\begin{lemma}
The eigenvectors of a real symmetric matrix~$A$ are real.
\end{lemma}
\begin{proof}
Suppose that $A$ is an $m \times m$ matrix and suppose that some eigenvalue~$\lambda$ 
is complex. Let $\vect{x}$ be a eigenvector corresponding to it, which can be a complex 
vector. Then 
\begin{equation}\label{pf:real_eigen1}
	A \vect{x} =  \lambda \vect{x}.
\end{equation}
Taking complex conjugates on each side, we get:
\begin{equation}\label{pf:real_eigen2}
	A \conj{\vect{x}} =  \conj{\lambda} \conj{\vect{x}}.
\end{equation}
Pre-multiply~(\ref{pf:real_eigen1})  by $\trans{\conj{\vect{x}}}$ 
and~(\ref{pf:real_eigen2}) by $\trans{\vect{x}}$ 
to obtain:
\begin{eqnarray}
	\trans{\conj{\vect{x}}} A \vect{x} =  \trans{\conj{\vect{x}}} \lambda \vect{x} \label{pf:real_eigen3}\\
	\trans{\vect{x} }A \conj{\vect{x}} =  \trans{\vect{x}} \conj{\lambda} \conj{\vect{x}} \label{pf:real_eigen4}.
\end{eqnarray}
Take the transpose of equation~(\ref{pf:real_eigen4}), we obtain: 
$\trans{\conj{\vect{x}}} A \vect{x} =  \trans{\conj{\vect{x}}} \conj{\lambda} \vect{x}$, 
where we made use of the fact that $A$ is symmetric. Now subtracting this equation from~(\ref{pf:real_eigen3}),
we obtain: 
\[
\trans{\vect{x}} (\lambda - \conj{\lambda}) \conj{\vect{x}} = 0.
\]
Since $\trans{\vect{x}} \conj{\vect{x}}$ is the sum of products of complex conjugates, 
it is not zero unless each component of $\vect{x}$ is zero. Since this is not the case 
($\vect{x}$ is an eigenvector), we must have $\lambda = \conj{\lambda}$ and hence $\lambda$
is real. 
\end{proof} 

\begin{example}
Consider the unit matrix 
\[
I_2 = \begin{pmatrix}
		1 & 0 \\
		0 & 1 \\
	 \end{pmatrix}.
\]
This has as eigenvalues $1$ and $1$, which are clearly not distinct. In general,
$I_m$ has $m$ eigenvalues all of which are $1$. Note that $I_m$ has rank~$m$.  
On the other hand, the matrix
\[
	\begin{pmatrix}
		1 & 1 \\
		1 & 1 \\
	 \end{pmatrix}
\]
has rank one and has two distinct eigenvalues~$0, 2$ which are the roots of 
the equation: $(1 - \lambda)^2 - 1 = 0$. Thus the rank of a matrix has 
nothing to do with the number of distinct eigenvalues. 
\end{example}

Now that we know that a real symmetric matrix has only real eigenvalues, what can we say 
about the corresponding eigenvectors? It turns out that the eigenvectors corresponding 
to the eigenvalues of a real symmetric matrix form an orthonormal basis for $\R^m$. 
We show this using a sequence of lemmas.
\begin{lemma}\label{lemma:orthogonal_eigenvectors}
The eigenvectors of a real symmetric matrix corresponding to distinct eigenvalues are 
mutually orthogonal.
\end{lemma}
\begin{proof}
Let $\lambda_i$ and $\lambda_j$ be two distinct eigenvalues of a real symmetric matrix $A$.
Let $\vect{x}_i$ and $\vect{x}_j$ be eigenvectors corresponding to these eigenvalues. 
Then $A \vect{x}_i = \lambda_i \vect{x}_i$ and $A \vect{x}_j = \lambda_j \vect{x}_j$. 
Pre-multiplying the first of these equations by $\trans{\vect{x}_j}$ and the second 
by $\trans{\vect{x}_i}$, we obtain: 
\begin{eqnarray}
	\trans{\vect{x}_j} A \vect{x}_i = \trans{\vect{x}_j} \lambda_i \vect{x}_i \label{pf:orthogonal1} \\
	\trans{\vect{x}_i} A \vect{x}_j = \trans{\vect{x}_i} \lambda_j \vect{x}_j \label{pf:orthogonal2}
\end{eqnarray}
Taking the transpose of the second of these equations, we obtain:
\begin{equation} \label{pf:orthogonal3}
	\trans{\vect{x}_j} A \vect{x}_i = \trans{\vect{x}_j} \lambda_j \vect{x}_i.
\end{equation}
Now the right-hand sides of equations~(\ref{pf:orthogonal1}) and~(\ref{pf:orthogonal2}) are 
identical. Hence $ \trans{\vect{x}_j} (\lambda_i - \lambda_j) \vect{x}_i = 0$ and since 
$\lambda_i \neq \lambda_j$, it must be that the vectors $\vect{x}_i$ and $\vect{x}_j$ 
are orthogonal.
\end{proof}

\begin{lemma}\label{lemma:eigenspace}
Let $A \in \R^{m \times m}$ and let $\lambda$ be an eigenvalue. 
The eigenvectors belonging to~$\lambda$ form a subspace $\eigenspace{A}{\lambda}$ 
of~$\R^m$.
\end{lemma}
\begin{proof}
Let $\vect{x}_1$ and $\vect{x}_2$ are two eigenvectors belonging to $\lambda$.
Then 
\[
    A (a \vect{x}_1 + b \vect{x}_2) = 
        a A \vect{x}_1 + b A \vect{x}_2 = 
        \lambda (a \vect{x}_1 + b \vect{x}_2).
\]
Thus any linear combination of the eigenvectors is also an eigenvector of $\lambda$,
showing that this set of vectors is indeed a subspace. 
\end{proof}

\begin{lemma}\label{lemma:alg_geom_mult}
Let $A \in \R^{m \times m}$ be a symmetric matrix. Then the algebraic multiplicity 
of each eigenvalue of $A$ equals its geometric multiplicity. 
\end{lemma}
\begin{proof}
Let $\lambda_i$ be an eigenvalue with algebraic multiplicity~$k$.  Suppose that
$\dim(\eigenspace{A}{\lambda_i}) = r$ and let $\{\vect{v}_1, \ldots, \vect{v}_r\}$ be an
orthonormal basis for $\eigenspace{A}{\lambda_i}$.  Extend this to an
orthonormal basis $\{\vect{v}_1, \ldots, \vect{v}_r, \vect{v}_{r + 1}, \ldots, \vect{v}_m\}$ 
for $\R^m$. 

Define $S = [\vect{v}_1, \ldots, \vect{v}_r, \vect{v}_{r + 1}, \ldots, \vect{v}_m]$ 
to be the $m \times m$ matrix whose columns are the basis vectors. 
Then $S$ is an orthonormal matrix and $S^{-1} = \trans{S}$. Consider the 
matrix $S^{-1} A S$. This is similar to $A$ and therefore has the 
same eigenvalues as $A$ including the same multiplicities. 

We may write $S^{-1} A S$ as:
\begin{align*}
    S^{-1} A S & = \begin{pmatrix} 
                        -\trans{\vect{v}_1}- \\
                         \vdots \\
                        -\trans{\vect{v}_m}- 
                   \end{pmatrix} 
                   A 
                   \begin{pmatrix}
                    \vect{v}_1 & \vect{v}_2 & \ldots & \vect{v}_m \\
                        |      &    |       &        &    |       
                   \end{pmatrix} \\
              & = \begin{pmatrix} 
                        -\trans{\vect{v}_1}- \\
                         \vdots \\
                        -\trans{\vect{v}_m}- 
                   \end{pmatrix} 
                   \begin{pmatrix}
                    \lambda_i \vect{v}_1 & \ldots & \lambda_i \vect{v}_r & A \vect{v}_{r + 1} & \ldots & A \vect{v}_m \\
                            |          &        &         |          &          |         &        &     |
                   \end{pmatrix} \\
             & = \left ( \begin{array}{c|c}
                    \lambda_i I_r         & \vect{0}_{r, m - r} \\ \hline
                    \vect{0}_{m - r, r} & \trans{C} A C 
                 \end{array} \right ),
\end{align*}
where $C = [\vect{v}_{r + 1}, \ldots, \vect{v}_m]$. The characteristic polynomials 
$\det (S^{-1} A S - \lambda I_m)$ and $\det (A - \lambda I_m)$ are identical 
and 
\[
   \det (S^{-1} A S - \lambda I_m) = (\lambda_i - \lambda)^r \det (\trans{C} A C - \lambda I_{m - r}). 
\]
Since we know that in the RHS, the term $\lambda_i - \lambda$ is raised to the $m$th power, 
we must have $r \leq m$. 
\end{proof}

Although we do not prove it here, the fact remains that for real symmetric
matrices, the algebraic multiplicity of an eigenvalue equals its geometric
multiplicity.  Since eigenvectors corresponding to two distinct eigenvalues of
a real symmetric matrix are orthogonal
(Lemma~\ref{lemma:orthogonal_eigenvectors}), by Lemmas~\ref{lemma:eigenspace}
and~\ref{lemma:alg_geom_mult}, the eigenspaces of distinct eigenvectors are
mutually orthogonal.
\begin{lemma}\label{lemma:orthonormal_basis}
Let $A \in \R^{m \times m}$ be a symmetric matrix. Then there exist an
orthonormal set of eigenvectors that form a basis for $\R^m$.
\end{lemma}
\begin{proof}
Let $\lambda_1, \ldots, \lambda_r$ be the distinct eigenvalues of $A$ with
algebraic multiplicities $m_1, \ldots, m_r$, respectively. Then $\sum_i m_i =
m$. Since the geometric multiplicity of an eigenvalue equals its algebraic
multiplicity, we have $\dim(\eigenspace{A}{\lambda_i}) = m_i$.  Since the
eigenspaces $\eigenspace{A}{\lambda_1}, \ldots, \eigenspace{A}{\lambda_r}$ are
mutually orthogonal, the union of their orthonormal bases is an independent set
of vectors of size~$m$.  Hence this forms a basis for $\R^m$.
\end{proof}

We can use Lemma~\ref{lemma:orthonormal_basis} to diagonalize a symmetric
matrix $A \in \R^{m \times m}$.  Let $\lambda_1, \ldots, \lambda_m$ be its
eigenvalues, in no particular order. These eigenvalues need not be distinct.
Let $\vect{u}_1, \ldots, \vect{u}_m$ be an orthonormal set of eigenvectors that
form a basis for $\R^m$. Define $P = [\vect{u}_1, \ldots, \vect{u}_m]$. Then
$P^{-1} A P = D$, where $D$ is a diagonal matrix whose main diagonal contains
the eigenvalues $\lambda_1, \ldots, \lambda_m$.

\begin{lemma}
Let $A \in \R^{m \times m}$ be a symmetric matrix. Then for any $\vect{x} \in R^m$,
\[
    \lambda_{\text{min}} \cdot \norm{\vect{x}}^2 \leq \trans{\vect{x}} A \vect{x} 
    \leq \lambda_{\text{max}} \cdot \norm{\vect{x}}^2,
\]
where $ \lambda_{\text{min}} $ and $ \lambda_{\text{max}} $ are a minimum 
and a maximum eigenvalue of $A$. 
\end{lemma}
\begin{proof}
Write $\vect{x} = \sum_{i = 1}^m a_i \vect{u}_i$, where $\vect{u}_1, \ldots, \vect{u}_m$
are an orthonormal set of eigenvectors that form a basis for $\R^m$. Then
\begin{align*}
\trans{\vect{x}} A \vect{x} & =  \left ( \sum_{i = 1}^m a_i \trans{\vect{u}_i} \right ) 
                                 \cdot A \cdot 
                                 \left ( \sum_{i = 1}^m a_i \vect{u}_i \right ) \\
& =  \left ( \sum_{i = 1}^m a_i \trans{\vect{u}_i} \right ) \cdot 
     \left ( \sum_{i = 1}^m a_i \lambda_i \vect{u}_i \right ) \\
& = \sum_{i = 1}^{m} a_i^2 \lambda_i \trans{\vect{u}_i} \vect{u}_i \\
& =  \sum_{i = 1}^{m} a_i^2 \lambda_i. 
\end{align*}
The last term lies clearly between $ \lambda_{\text{min}} \cdot \norm{\vect{x}}^2 $
and $\lambda_{\text{max}} \cdot \norm{\vect{x}}^2$. 
\end{proof}

We can now state the characterization of quadractic forms.
\begin{theorem}
A real symmetric matrix $A$ is 
\begin{enumerate}
    \item positive definite iff all its eigenvalues are strictly positive;
    \item negative definite iff all its eigenvalues are strictly negative;
    \item indefinite iff some of its eigenvalues are positive and some are negative.
\end{enumerate}
\end{theorem}
