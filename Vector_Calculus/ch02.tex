\chapter{Excercises on Singular Value Decompositions}

\begin{exercise}
Suppose $T \in \linspace{V}$ is self-adjoint. Prove that the singular values of $T$
are the absolute values of the eigenvalues of $T$ repeated appropriately. This is
\emph{not} true for operators that are not self-adjoint.
\end{exercise}
\begin{solution}
Suppose that $T$ is self-adjoint with eigenvalues $\lambda_1, \ldots, \lambda_n$.
Then $\adj{T}T$ is positive and has eigenvalues
$\conj{\lambda}_1 \lambda_1, \ldots, \conj{\lambda}_n \lambda_n$. The eigenvalues
of $\sqrt{\adj{T}T}$ are the positive square roots of the eigenvalues of $\adj{T} T$.
Hence the singular values of $T$ are $|\lambda_1|, \ldots, |\lambda_n|$.
\end{solution}

\begin{exercise}
Suppose $T \in \linspace{V}$. Prove that $T$ and $\adj{T}$ have the same singular
values.
\end{exercise}
\begin{solution}
For any $T \in \linspace{V}$, the operators $\adj{T} T$ and $T \adj{T}$ are both
positive. It is sufficient to show that both $\adj{T} T$ and $T \adj{T}$ have the
same eigenvalues with the same multiplicities.

Using the Polar Decomposition Theorem, we may write $T$ as $S \sqrt{\adj{T} T}$, for
some isometry~$S$. Now $\sqrt{\adj{T} T}$ is positive and hence self-adjoint. Thus
\[
    \adj{T} = \adj{(S \sqrt{\adj{T} T})} = \sqrt{\adj{T} T} \adj{S} =
              \sqrt{\adj{T} T} \inv{S},
\]
and $T \adj{T} = S \sqrt{\adj{T} T} \sqrt{\adj{T} T} \inv{S} = S \adj{T} T \inv{S}$.
Thus $T \adj{T}$ and $\adj{T} T$ are similar matrices and therefore have the same
eigenvalues with the same multiplicities.
\end{solution}

\begin{exercise}
Suppose $T_1, T_2, S \in \linspace{V}$ are such that $S$ is invertible and
$T_1 = S T_2 \inv{S}$, then $T_1$ and $T_2$ have the same eigenvalues with the
same multiplicities.
\end{exercise}
\begin{solution}
Let $u \in V$ be an eigenvector of $T_2$ corresponding to the eigenvalue $\lambda$.
Define $v = S u$ so that $\inv{S} v = u$. Then
\begin{align*}
    T_1 v & = S T_2 \inv{S} v \\
          & = S T_2 u \\
          & = \lambda S u \\
          & = \lambda v.
\end{align*}
Hence $\lambda$ is an eigenvalue of $T_1$. Thus every eigenvalue of $T_2$ is an
eigenvalue of $T_1$. A symmetric argument shows that every eigenvalue of $T_1$
is an egienvalue of $T_2$. Thus $T_1$ and $T_2$ have the same set of eigenvalues.

Since $S$ is invertible,
$\dim S (\eigenspace{T_2}{\lambda}) = \dim \eigenspace{T_2}{\lambda}$ and for all
$v \in S ( \eigenspace{T_2}{\lambda} )$ we have that $T_1 v = \lambda v$. This shows
that $\dim \eigenspace{T_1}{\lambda} \leq \dim \eigenspace{T_2}{\lambda}$.
A symmetric argument will show that
$\dim \eigenspace{T_2}{\lambda} \leq \dim \eigenspace{T_1}{\lambda}$. Hence
$\dim \eigenspace{T_1}{\lambda} = \dim \eigenspace{T_2}{\lambda}$. Thus each
eigenvalue of $T_1$ and $T_2$ has the same multiplicity.
\end{solution}

\begin{exercise}
Let $T \in \linspace{V}$. Prove that $T$ is singular iff $0$ is not a singular
value of $T$.
\end{exercise}
\begin{solution}
By the Polar Decomposition Theorem, $T = S \sqrt{\adj{T} T}$, for some isometry~$S$.
Since $S$ is an isometry, it is invertible and hence $T$ is invertible iff
$\sqrt{\adj{T} T}$ is. The latter is a positive (and hence self-adjoint) operator.
By the Complex Spectral Theorem, there exists an orthonormal basis of $V$
consisting of the eigenvectors of $\sqrt{\adj{T} T}$. The matrix of
$\sqrt{\adj{T} T}$ w.r.t this orthonormal basis is a diagonal matrix whose main
diagonal consists of the corresponding eigenvalues. Hence $\sqrt{\adj{T} T}$
is invertible iff it does not have $0$ as one of its eigenvalues iff
$0$ is not a singular value of $T$.
\end{solution}

\begin{exercise}
Suppose that $T \in \linspace{V}$. Prove that $\dim \range T$ equals the number
of non-zero singular values of $T$.
\end{exercise}
\begin{solution}
Let $\dim V = n$ and let $\lambda_1, \ldots, \lambda_m$ be the distinct eigenvalues
of $\sqrt{\adj{T} T}$. If $u$ and $v$ are eigenvectors corresponding to distinct
eigenvalues $\lambda_i$ and $\lambda_j$, respectively, then $\innerprod{u}{v} = 0$.
One consequence of this is that
\[
    V = \eigenspace{\sqrt{\adj{T} T}}{\lambda_1} \oplus \cdots \oplus
        \eigenspace{\sqrt{\adj{T} T}}{\lambda_m}.
\]
First consider the case that none of the eigenvalues of $\sqrt{\adj{T} T}$ are $0$.
In this case, $T$ has $n$ non-zero singular values.
Let $v \in V$ be non-zero. Because of the previous identity, there exists $\lambda$
such that $\sqrt{\adj{T} T} v = \lambda v$. Moreover, by the Polar Decomposition
Theorem, $T = S \sqrt{\adj{T} T}$, where $S$ is an isometry and hence invertible.
Thus,
\[
    Tv = S \sqrt{\adj{T} T} v = \lambda S v \neq 0.
\]
This shows that $0$ is the only vector in $\nullspace T$. Since
$\dim V = \dim \nullspace T + \dim \range T$, we have that $\dim \range T = n$,
the number of non-zero singular values of $T$.

Next consider the case when one of the eigenvalues of $\sqrt{\adj{T} T}$ is $0$. Wlog
assume that $\lambda_1 = 0$ and that $\dim \eigenspace{\sqrt{\adj{T} T}}{\lambda_1} = k$.
Then $T$ has $n - k$ non-zero singular values.
Now repeating the argument of the previous paragraph, if
$v \in \eigenspace{\sqrt{\adj{T} T}}{\lambda_2} \oplus \cdots \oplus
\eigenspace{\sqrt{\adj{T} T}}{\lambda_m}$ is non-zero then $Tv \neq 0$. Hence
$\dim \range T \geq n - k$. Since for all
$v \in \eigenspace{\sqrt{\adj{T} T}}{\lambda_1}$, we have that $T v = 0$, we conclude
that $\dim \nullspace T \geq k$. But $\dim V = \dim \nullspace T + \dim \range T$
and so we must have $\dim \nullspace T = k$ and $\dim \range T = n - k$. This
concludes the proof.
\end{solution}

\begin{exercise}
Suppose that $S \in \linspace{V}$. Prove that $S$ is an isometry iff all its singular
values are equal to $1$.
\end{exercise}
\begin{solution}
First suppose that $S$ is an isometry. Then $\adj{S} = \inv{S}$ and
$\sqrt{\adj{S} S} = I$. Hence the singular values of $S$ are the eigenvalues of the
identity operator, which are all $1$.

Next suppose that $S$ is an operator whose singular values are all equal to $1$.
Then the operator $\sqrt{\adj{S} S}$ has only $1$ as its eigenvalue. Since
$\sqrt{\adj{S} S}$ is positive (and hence self-adjoint), by the Spectral Theorem
there exists an orthonormal basis $(e_1, \ldots, e_n)$ of $V$ which are the
eigenvectors of $\sqrt{\adj{S} S}$. This implies that for $1 \leq j \leq n$, we
have $\sqrt{\adj{S} S} e_j = e_j$. Hence $\sqrt{\adj{S} S} = I$, the identity
operator. This, in turn, implies that $\adj{S} S = I$ which is true iff $S$ is
an isometry.
\end{solution}

\begin{exercise}
Suppose that $T \in \linspace{V}$. Let $s_{\min}$ and $s_{\max}$ denote, respectively,
the smallest and largest singular values of $T$. Show that
\begin{enumerate}
    \item $s_{\min} \cdot \norm{v} \leq \norm{T v} \leq s_{\max} \cdot \norm{v}$
        for all $v \in V$;
    \item $s_{\min} \leq |\lambda| \leq s_{\max}$ for every eigenvalue $\lambda$ of
        $T$.
\end{enumerate}
\end{exercise}
\begin{solution}
Let $s_1, \ldots, s_n$ be the singular values of $T$ ane let $v \in V$. By the
Singular Value Decomposition, there exist orthonormal bases $(e_1, \ldots, e_n)$
and $(f_1, \ldots, f_n)$ such that
\[
    T v = s_1 \innerprod{v}{e_1} f_1 + \cdots + \innerprod{v}{e_n} f_n.
\]
Since $(f_1, \ldots, f_n)$ is an orthonormal basis,
$\norm{T v}^2 = \sum_{j = 1}^{n} | s_j \innerprod{v}{e_j}|^2 =
\sum_{j = 1}^{n} s_j^2 |\innerprod{v}{e_j}|^2$. The last equality follows because
all the singular values of $T$ are non-negative. We can upper and lower
bound the quantity $\norm{T v}^2$ as follows:
\[
    s_{\min}^2  \cdot \sum_{j = 1}^{n} |\innerprod{v}{e_j}|^2 \leq \norm{T v}^2
    \leq  s_{\max}^2  \cdot \sum_{j = 1}^{n} |\innerprod{v}{e_j}|^2.
\]
Since $(e_1, \ldots, e_n)$ is orthonormal,
$\sum_{j = 1}^{n} |\innerprod{v}{e_j}|^2 = \norm{v}^2$. Now taking square roots, we
obtain the desired inequality.

If $v$ is an eigenvector of $T$ with eigenvalue $\lambda$,
$\norm{T v} = |\lambda | \cdot \norm{v}$. We thus have:
\[
    s_{\min} \cdot \norm{v} \leq |\lambda | \cdot \norm{v}
                            \leq s_{\max} \cdot \norm{v};
\]
since $\norm{v} \neq 0$, dividing throughout by $\norm{v}$, we obtain
$s_{\min} \leq |\lambda| \leq s_{\max}$.
\end{solution}

\begin{exercise}
Suppose that $T \in \linspace{V}$ has a Singular Value Decomposition given by
\[
    \forall \ v \in V \colon T v =
    s_1 \innerprod{v}{e_1} f_1 + \cdots + s_n \innerprod{v}{e_n} f_n,
\]
where $s_1, \ldots, s_n$ are the singular values of $T$. Prove that:
\begin{enumerate}
    \item $\adj{T} v = s_1 \innerprod{v}{f_1} e_1 + \cdots + s_n \innerprod{v}{f_n} e_n$;
    \item $\adj{T} T v = s_1^2 \innerprod{v}{e_1} e_1 + \cdots + s_n^2 \innerprod{v}{e_n} e_n$
    \item $\sqrt{\adj{T} T} v = s_1 \innerprod{v}{e_1} e_1 + \cdots + s_n \innerprod{v}{e_n} e_n$
\end{enumerate}
\end{exercise}
\begin{solution}
Let $T = S \sqrt{\adj{T} T}$ be the polar decomposition of $T$ that is associated
with the singular value decomposition given above. Then $(e_1, \ldots, e_n)$ are
the eigenvectors of $\sqrt{\adj{T} T}$ and $\sqrt{\adj{T} T} e_j = s_j e_j$ for
all $1 \leq j \leq n$. Moreover $S e_j = f_j$ for all $1 \leq j \leq n$.

We can express $\adj{T}$ as $\adj{T} = \sqrt{\adj{T} T} \adj{S} = \sqrt{\adj{T} T} \inv{S}$.
Let $v \in V$ be any vector. Then
\begin{align*}
    \adj{T} v & = \sqrt{\adj{T} T} \inv{S} (\innerprod{v}{f_1} f_1 + \cdots + \innerprod{v}{f_n} f_n) \\
              & = \sqrt{\adj{T} T} (\innerprod{v}{f_1} e_1 + \cdots + \innerprod{v}{f_n} e_n) \\
              & = s_1 \innerprod{v}{f_1} e_1 + \cdots + s_n \innerprod{v}{f_n} e_n.
\end{align*}
This proves~$(1)$.

Next we may write $\adj{T} T v$ as:
\begin{align*}
    \adj{T} T v & = \sqrt{\adj{T} T} \inv{S} T v \\
                & = \sqrt{\adj{T} T} \inv{S} (s_1 \innerprod{v}{e_1} f_1 + \cdots + s_n \innerprod{v}{e_n} f_n) \\
                & = \sqrt{\adj{T} T} (s_1 \innerprod{v}{e_1} e_1 + \cdots + s_n \innerprod{v}{e_n} e_n) \\
                & = s_1^2 \innerprod{v}{e_1} e_1 + \cdots + s_n^2 \innerprod{v}{e_n} e_n.
\end{align*}
This proves~$(2)$.

Since $\sqrt{\adj{T} T} e_j = s_j e_j$, we may write $\sqrt{\adj{T} T} v$ as:
\begin{align*}
    \sqrt{\adj{T} T} v & = \sqrt{\adj{T} T} (\innerprod{v}{e_1} e_1 + \cdots + \innerprod{v}{e_n} e_n) \\
                       & = s_1 \innerprod{v}{e_1} e_1 + \cdots + s_n \innerprod{v}{e_n} e_n.
\end{align*}
And this proves~$(3)$.
\end{solution}

\begin{exercise}
Suppose that $T \in \linspace{V}$. Prove that $T$ is invertible iff there exists
a unique isometry $S$ such that $T = S \sqrt{\adj{T} T}$.
\end{exercise}
\begin{solution}
First suppose that $T$ is invertible and that $T = S_1 \sqrt{\adj{T} T} =
S_2 \sqrt{\adj{T} T}$, where $_S1$ and $S_2$ are isometries. Since we assumed $T$
to be invertible, it must be that $\sqrt{\adj{T} T}$ is also invertible. The
operators $S_1$ and $S_2$ being isometries are invertible. Thus
$\inv{T} = \inv{\sqrt{\adj{T} T}} \inv{S_1}$,
and hence
\[
    I = T \inv{T} = S_2 \sqrt{\adj{T} T} \inv{\sqrt{\adj{T} T}} \inv{S_1} = S_2 \inv{S_1}.
\]
From the last equation we obtain that $S_1 = S_2$. Hence the isomtery appearing
in the polar decomposition of $T$ is unique.

\end{solution}
