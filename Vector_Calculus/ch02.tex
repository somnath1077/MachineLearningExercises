\chapter{Excercises on Singular Value Decompositions}

\begin{exercise}
Suppose $T \in \linspace{V}$ is self-adjoint. Prove that the singular values of $T$
are the absolute values of the eigenvalues of $T$ repeated appropriately. This is
\emph{not} true for operators that are not self-adjoint.
\end{exercise}
\begin{solution}
Suppose that $T$ is self-adjoint with eigenvalues $\lambda_1, \ldots, \lambda_n$.
Then $\adj{T}T$ is positive and has eigenvalues
$\conj{\lambda}_1 \lambda_1, \ldots, \conj{\lambda}_n \lambda_n$. The eigenvalues
of $\sqrt{\adj{T}T}$ are the positive square roots of the eigenvalues of $\adj{T} T$.
Hence the singular values of $T$ are $|\lambda_1|, \ldots, |\lambda_n|$.
\end{solution}

\begin{exercise}
Suppose $T \in \linspace{V}$. Prove that $T$ and $\adj{T}$ have the same singular
values.
\end{exercise}
\begin{solution}
For any $T \in \linspace{V}$, the operators $\adj{T} T$ and $T \adj{T}$ are both
positive. It is sufficient to show that both $\adj{T} T$ and $T \adj{T}$ have the
same eigenvalues with the same multiplicities.

Using the Polar Decomposition Theorem, we may write $T$ as $S \sqrt{\adj{T} T}$, for
some isometry~$S$. Now $\sqrt{\adj{T} T}$ is positive and hence self-adjoint. Thus
\[
    \adj{T} = \adj{(S \sqrt{\adj{T} T})} = \sqrt{\adj{T} T} \adj{S} =
              \sqrt{\adj{T} T} \inv{S},
\]
and $T \adj{T} = S \sqrt{\adj{T} T} \sqrt{\adj{T} T} \inv{S} = S \adj{T} T \inv{S}$.
Thus $T \adj{T}$ and $\adj{T} T$ are similar matrices and therefore have the same
eigenvalues with the same multiplicities.
\end{solution}

\begin{exercise}
Suppose $T_1, T_2, S \in \linspace{V}$ are such that $S$ is invertible and
$T_1 = S T_2 \inv{S}$, then $T_1$ and $T_2$ have the same eigenvalues with the
same multiplicities.
\end{exercise}
\begin{solution}
Let $u \in V$ be an eigenvector of $T_2$ corresponding to the eigenvalue $\lambda$.
Define $v = S u$ so that $\inv{S} v = u$. Then
\begin{align*}
    T_1 v & = S T_2 \inv{S} v \\
          & = S T_2 u \\
          & = \lambda S u \\
          & = \lambda v.
\end{align*}
Hence $\lambda$ is an eigenvalue of $T_1$. Thus every eigenvalue of $T_2$ is an
eigenvalue of $T_1$. A symmetric argument shows that every eigenvalue of $T_1$
is an egienvalue of $T_2$. Thus $T_1$ and $T_2$ have the same set of eigenvalues.

Since $S$ is invertible,
$\dim S (\eigenspace{T_2}{\lambda}) = \dim \eigenspace{T_2}{\lambda}$ and for all
$v \in S ( \eigenspace{T_2}{\lambda} )$ we have that $T_1 v = \lambda v$. This shows
that $\dim \eigenspace{T_1}{\lambda} \leq \dim \eigenspace{T_2}{\lambda}$.
A symmetric argument will show that
$\dim \eigenspace{T_2}{\lambda} \leq \dim \eigenspace{T_1}{\lambda}$. Hence
$\dim \eigenspace{T_1}{\lambda} = \dim \eigenspace{T_2}{\lambda}$. Thus each
eigenvalue of $T_1$ and $T_2$ has the same multiplicity.
\end{solution}

\begin{exercise}
Let $T \in \linspace{V}$. Prove that $T$ is singular iff $0$ is not a singular
value of $T$.
\end{exercise}
\begin{solution}
By the Polar Decomposition Theorem, $T = S \sqrt{\adj{T} T}$, for some isometry~$S$.
Since $S$ is an isometry, it is invertible and hence $T$ is invertible iff
$\sqrt{\adj{T} T}$ is. The latter is a positive (and hence self-adjoint) operator.
By the Complex Spectral Theorem, there exists an orthonormal basis of $V$
consisting of the eigenvectors of $\sqrt{\adj{T} T}$. The matrix of
$\sqrt{\adj{T} T}$ w.r.t this orthonormal basis is a diagonal matrix whose main
diagonal consists of the corresponding eigenvalues. Hence $\sqrt{\adj{T} T}$
is invertible iff it does not have $0$ as one of its eigenvalues iff
$0$ is not a singular value of $T$.
\end{solution}

\begin{exercise}
Suppose that $T \in \linspace{V}$. Prove that $\dim \range T$ equals the number
of non-zero singular values of $T$.
\end{exercise}
\begin{solution}
Let $\dim V = n$ and let $\lambda_1, \ldots, \lambda_m$ be the distinct eigenvalues
of $\sqrt{\adj{T} T}$. If $u$ and $v$ are eigenvectors corresponding to distinct
eigenvalues $\lambda_i$ and $\lambda_j$, respectively, then $\innerprod{u}{v} = 0$.
One consequence of this is that
\[
    V = \eigenspace{\sqrt{\adj{T} T}}{\lambda_1} \oplus \cdots \oplus
        \eigenspace{\sqrt{\adj{T} T}}{\lambda_m}.
\]
First consider the case that none of the eigenvalues of $\sqrt{\adj{T} T}$ are $0$.
In this case, $T$ has $n$ non-zero singular values.
Let $v \in V$ be non-zero. Because of the previous identity, there exists $\lambda$
such that $\sqrt{\adj{T} T} v = \lambda v$. Moreover, by the Polar Decomposition
Theorem, $T = S \sqrt{\adj{T} T}$, where $S$ is an isometry and hence invertible.
Thus,
\[
    Tv = S \sqrt{\adj{T} T} v = \lambda S v \neq 0.
\]
This shows that $0$ is the only vector in $\nullspace T$. Since
$\dim V = \dim \nullspace T + \dim \range T$, we have that $\dim \range T = n$,
the number of non-zero singular values of $T$.

Next consider the case when one of the eigenvalues of $\sqrt{\adj{T} T}$ is $0$. Wlog
assume that $\lambda_1 = 0$ and that $\dim \eigenspace{\sqrt{\adj{T} T}}{\lambda_1} = k$.
Then $T$ has $n - k$ non-zero singular values.
Now repeating the argument of the previous paragraph, if
$v \in \eigenspace{\sqrt{\adj{T} T}}{\lambda_2} \oplus \cdots \oplus
\eigenspace{\sqrt{\adj{T} T}}{\lambda_m}$ is non-zero then $Tv \neq 0$. Hence
$\dim \range T \geq n - k$. Since for all
$v \in \eigenspace{\sqrt{\adj{T} T}}{\lambda_1}$, we have that $T v = 0$, we conclude
that $\dim \nullspace T \geq k$. But $\dim V = \dim \nullspace T + \dim \range T$
and so we must have $\dim \nullspace T = k$ and $\dim \range T = n - k$. This
concludes the proof.
\end{solution}
