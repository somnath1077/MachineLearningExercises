\chapter{Excercises on Singular Value Decompositions}

\begin{exercise}
suppose $T \in \linspace{V}$ is self-adjoint. Prove that the singular values of $T$
are the absolute values of the eigenvalues of $T$ repeated appropriately. This is
\emph{not} true for operators that are not self-adjoint.
\end{exercise}
\begin{solution}
Suppose that $T$ is self-adjoint with eigenvalues $\lambda_1, \ldots, \lambda_n$.
Then $\adj{T}T$ is positive and has eigenvalues
$\conj{\lambda}_1 \lambda_1, \ldots, \conj{\lambda}_n \lambda_n$. The eigenvalues
of $\sqrt{\adj{T}T}$ are the positive square roots of the eigenvalues of $\adj{T} T$.
Hence the singular values of $T$ are $|\lambda_1|, \ldots, |\lambda_n|$.
\end{solution}

\begin{exercise}
Suppose $T \in \linspace{V}$. Prove that $T$ and $\adj{T}$ have the same singular
values.
\end{exercise}
