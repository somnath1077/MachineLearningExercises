\chapter{Excercises on Singular Value Decompositions}

\begin{exercise}
Suppose $T \in \linspace{V}$ is self-adjoint. Prove that the singular values of $T$
are the absolute values of the eigenvalues of $T$ repeated appropriately. This is
\emph{not} true for operators that are not self-adjoint.
\end{exercise}
\begin{solution}
Suppose that $T$ is self-adjoint with eigenvalues $\lambda_1, \ldots, \lambda_n$.
Then $\adj{T}T$ is positive and has eigenvalues
$\conj{\lambda}_1 \lambda_1, \ldots, \conj{\lambda}_n \lambda_n$. The eigenvalues
of $\sqrt{\adj{T}T}$ are the positive square roots of the eigenvalues of $\adj{T} T$.
Hence the singular values of $T$ are $|\lambda_1|, \ldots, |\lambda_n|$.
\end{solution}

\begin{exercise}
Suppose $T \in \linspace{V}$. Prove that $T$ and $\adj{T}$ have the same singular
values.
\end{exercise}
\begin{solution}
For any $T \in \linspace{V}$, the operators $\adj{T} T$ and $T \adj{T}$ are both
positive. It is sufficient to show that both $\adj{T} T$ and $T \adj{T}$ have the
same eigenvalues with the same multiplicities.

Using the Polar Decomposition Theorem, we may write $T$ as $S \sqrt{\adj{T} T}$, for
some isometry~$S$. Now $\sqrt{\adj{T} T}$ is positive and hence self-adjoint. Thus
\[
    \adj{T} = \adj{(S \sqrt{\adj{T} T})} = \sqrt{\adj{T} T} \adj{S} =
              \sqrt{\adj{T} T} \inv{S},
\]
and $T \adj{T} = S \sqrt{\adj{T} T} \sqrt{\adj{T} T} \inv{S} = S \adj{T} T \inv{S}$.
Thus $T \adj{T}$ and $\adj{T} T$ are similar matrices and therefore have the same
eigenvalues with the same multiplicities.
\end{solution}

\begin{exercise}
Suppose $T_1, T_2, S \in \linspace{V}$ are such that $S$ is invertible and
$T_1 = S T_2 \inv{S}$, then $T_1$ and $T_2$ have the same eigenvalues with the
same multiplicities.
\end{exercise}
\begin{solution}
Let $u \in V$ be an eigenvector of $T_2$ corresponding to the eigenvalue $\lambda$.
Define $v = S u$ so that $\inv{S} v = u$. Then
\begin{align*}
    T_1 v & = S T_2 \inv{S} v \\
          & = S T_2 u \\
          & = \lambda S u \\
          & = \lambda v.
\end{align*}
Hence $\lambda$ is an eigenvalue of $T_1$. Thus every eigenvalue of $T_2$ is an
eigenvalue of $T_1$. A symmetric argument shows that every eigenvalue of $T_1$
is an egienvalue of $T_2$. Thus $T_1$ and $T_2$ have the same set of eigenvalues.

Since $S$ is invertible,
$\dim S (\eigenspace{T_2}{\lambda}) = \dim \eigenspace{T_2}{\lambda}$ and for all
$v \in S ( \eigenspace{T_2}{\lambda} )$ we have that $T_1 v = \lambda v$. This shows
that $\dim \eigenspace{T_1}{\lambda} \leq \dim \eigenspace{T_2}{\lambda}$.
A symmetric argument will show that
$\dim \eigenspace{T_2}{\lambda} \leq \dim \eigenspace{T_1}{\lambda}$. Hence
$\dim \eigenspace{T_1}{\lambda} = \dim \eigenspace{T_2}{\lambda}$. Thus each
eigenvalue of $T_1$ and $T_2$ has the same multiplicity.
\end{solution}
