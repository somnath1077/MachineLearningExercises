\chapter{The No-Free-Lunch Theorem}

\section*{Notes on Chapter 5}

Consider a binary classification task on a domain $\dom$. Assume for the time
being that $\dom$ is finite. In this case, the set $\hypclass$ of all functions
from $\dom \rightarrow \{0, 1\}$ is finite and is hence PAC-learnable with
sample complexity $\leq \frac{\log (|\hypclass| / \delta)}{\epsilon}$.  Since
$|\hypclass| = 2^{|\dom|}$, the sample complexity is $\frac{|\dom| + \log (1 /
\delta)}{\epsilon} = O(|\dom|)$.

Let us suppose that $A$ is a learning algorithm for the task of binary
classification w.r.t $\zeroone$ loss over the domain $\dom$. Furthermore, assume
that $A$ has no access to any prior knowledge in the sense that the hypothesis
class from which it chooses its hypotheses is the set of all functions from
$\dom \rightarrow \{0, 1\}$.
The first question is what happens wrt PAC-learnability in this situation when
we restrict the sample size? The No-Free-Lunch theorem shows that there exists
a distribution $\dist$ over $\dom \times \{0, 1\}$ and a labelling function $f
\colon \dom \rightarrow \{0, 1\}$ that learners who are  constrained to use at
most $|\dom| / 2$ training examples ``cannot learn.'' However, given specific prior
knowledge in the form of a hypothesis class that contains this function~$f$, the
$\ERM$ algorithm is a successful learner.


There is another way to interpret the No-Free-Lunch theorem: if the domain
$\dom$ is \emph{infinite}, then the set of all functions from $\dom$ to
$\{0, 1\}$ is not PAC-learnable no matter what the sample size.

Thus the No-Free-Lunch theorem has two interpretations. Firstly, it shows that there
is no universal learner in the sense of a learning algorithm that succeeds on
learning tasks without prior information. Secondly, it shows that arbitrary hypothesis
classes are not PAC-learnable.


\begin{theorem}
\label{thm:no_free_lunch}
Consider the task of binary classification over the domain $\dom$ wrt the $\zeroone$
loss function. Let $A$ be a learning algorithm that is constrained to use at
most $m \leq |\dom| / 2$ training examples. Then there exist
a function $f \colon \dom \rightarrow \{0, 1\}$ and a distribution
$\dist$ over $\dom \times \{0, 1\}$ such that
\begin{enumerate}
    \item $L_{\dist} (f) = 0$
    \item with probability of at least $1/7$ over the choice of training examples
        chosen iid from $\dist^{m}$, we have that $L_{\dist} (A(S)) \geq 1/ 8$.
\end{enumerate}
\end{theorem}

\begin{corollary}
Let $\dom$ be an infinite domain set and let $\hypclass$ be the set of all boolean
functions on $\dom$. Then $\hypclass$ is not PAC-learnable.
\end{corollary}

\section*{Exercise 5.1}

As the hint in the exercise suggests, let $\theta$ be a random variable that takes
on values in the range $[0, 1]$ with expectation $\Expone{\theta} \geq 1 / 4$. We
want to show that $\Prone{\theta \geq 1 / 8} \geq 1 / 7$.

We start with Markov's inequality: for any nonnegative random variable $X$ and $a > 0$,
\[
    \Prone{X \geq a} \leq \frac{\Expone{X}}{a}.
\]
This doesn't quite work when we substitute $\theta = X$ and $a = 1 / 8$. The
trick here lies in observing that $\theta$ is bounded from above by $1$, and
hence, if we define $\xi = 1 - \theta$ then $\xi$ is nonnegative and we can use
Markov's inequality on $\xi$. Note that $\Expone{\xi} =  1 - \Expone{\theta}$,
and hence by Markov's inequality,
\begin{align*}
    \Prone{\xi \geq a} & \leq \frac{\Expone{\xi}}{a} \\
    1 -  \Prone{\xi \geq a} & \geq 1 - \frac{\Expone{\xi}}{a} \\
    \Prone{\xi < a} & \geq 1 - \frac{1 - \Expone{\theta}}{a} \\
    \Prone{1 - \theta < a} & \geq \frac{a - 1}{a} + \frac{\Expone{\theta}}{a}
\end{align*}
At this point, we use the fact that $\Expone{\theta} \geq 1 / 4$ to obtain:
$\Prone{\theta > 1 - a} \geq \frac{a - 1}{a} + \frac{1}{4a}$. Now if we substitute
$1 - a =  1 / 8$, or $a = 7 / 8$, then we obtain:
\[
    \Prone{\theta > 1 / 8} \geq 1 / 7.
\]

\section*{Exercise 5.2}

The first algorithm, the one that picks only blood pressure and the BMI as
features, is simpler in the sense that the hypothesis class to be learned in
simpler. We would expect that this algorithm has a higher bias but a lower
variance when compared to the second algorithm which is more feature rich. The second
algorithm would probably explain the conditions of a heart attack better as it
includes relevant features such as age and the level of physical activity into
account. We would expect the second algorithm to have a lower bias but a higher
variance because there may be a tendency to overfit on any given sample.

Since the sample complexity is higher for a more complicated hypothesis class,
if the sample size is ``small,'' then we might want to choose the first algorithm.
If sample size is not a problem, then the second algorithm is probably better.
