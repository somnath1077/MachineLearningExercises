\chapter{A Formal Learning Model}

\section*{Notes on Chapter 3}

The main concept introduced here is that of agnostic PAC learnability. It helps to
review the definitions of both PAC learnability with the realizability assumption
and that of agnostic PAC learnability.

\begin{definition}[PAC Learnability]
Fix a domain $\dom$, a range $\range$ and let $\hypclass$ be a set of functions
from $\dom \rightarrow \range$. The class $\hypclass$ is PAC learnable if there
exists a function $\samplecomp \colon (0, 1) \times (0, 1) \rightarrow \Nat$
and a learning algorithm $\algo$ such that the following holds: for all
$\epsilon, \delta \in (0, 1)$, all labeling functions $f \colon \dom
\rightarrow \range$ and all distributions $\dist$ over $\dom$ such that
$\hypclass$ is realizable wrt $\dist$ and $f$, if $\algo$ is presented with a
sample of at least $\SampleComp{\epsilon}{\delta}$ examples drawn iid from
$\dist$, $\algo$ returns a hypothesis $h_S$ such that
\[
     \Prtwo{S \sim \dist^m}{L_{\dist, f} (h_S) \leq \epsilon}
        > 1 - \delta.
\]
\end{definition}

In the definition above, the sample complexity function $\samplecomp$ must work
for all possible labeling functions $f$ and distributions $\dist$ as long as
$\hypclass$ is realizable (wrt this labeling function and distribution). A
learning task is completely specified by $(\dom, \range, f, \dist)$.
Intuitively, what this definition states is that a hypothesis class is PAC
learnable if there exists a learner which when given a sufficiently large
number of training examples can approximate the true labeling function $f$ with
high probability for all learning tasks on the domain $\dom \times \range$ that
satisfy the realizability condition.

\begin{definition}[Agnostic PAC Learnability]
A class $\hypclass$ is PAC learnable if there exists a function $\samplecomp
\colon (0, 1) \times (0, 1) \rightarrow \Nat$ and a learning algorithm $\algo$
such that the following holds: for all $\epsilon, \delta \in (0, 1)$ and all
distributions $\dist$ over $\dom \times \range$ such that, if $\algo$ is
presented with a sample of at least $\SampleComp{\epsilon}{\delta}$ examples
drawn iid from $\dist$, $\algo$ returns a hypothesis $h_S$ such that
\[
   \Prtwo{S \sim \dist^m}{L_{\dist} (h_S) \leq
        \min_{h' \in \hypclass} L_{\dist} (h') + \epsilon} > 1 - \delta.
\]
\end{definition}
In the agnostic setting, a learning task is completely specified by the triplet
$(\dom, \range, \dist)$ and a hypothesis class is agnostic PAC learnable if
there exists a leraner which when given a sufficiently many training examples
outputs a ``good enough'' hypothesis with high probability. The goodness of the
hypothesis is measured with respect to the best hypothesis of the class $\hypclass$.

Agnostic PAC learnability presents a \emph{stronger} requirement as one must
be able to learn \emph{any} distibution as distinct from PAC learnability where
one must be able to learn distibutions for which the hypothesis class is realizable.
As such, agnostic PAC learnability implies PAC learnability.

\section*{Exercise 3.1}

Let $\SampleComp{\epsilon}{\delta}$ be the sample complexity of a PAC-learnable
hypothesis class $\mathcal{H}$ for a binary classification task. For a fixed
$\delta$, let $0 < \epsilon_1 \leq \epsilon_2 < 1$ and suppose that
$\SampleComp{\epsilon_1}{\delta} < \SampleComp{\epsilon_2}{\delta}$. Then when
running the learning algorithm on $\SampleComp{\epsilon_1}{\delta}$ i.i.d
examples, we obtain a hypothesis $h$, which with probability at least $1 -
\delta$ has a true error $\pacerror{h} \leq \epsilon_1 \leq \epsilon_2$. This
implies that for the $(\epsilon_2, \delta)$ combination of parameters, we can
bound the true error of $h$ by $\epsilon_2$ by using a smaller number of i.i.d
examples than $\SampleComp{\epsilon_2}{\delta}$. This contradicts the
minimality of the sample complexity function. Hence we must have
$\SampleComp{\epsilon_1}{\delta} \geq \SampleComp{\epsilon_2}{\delta}$.

Next suppose that $0 < \delta_1 \leq \delta_2 < 1$ and that
$\SampleComp{\epsilon}{\delta_1} < \SampleComp{\epsilon}{\delta_2}$, where
$\epsilon$ is fixed in advance. Then with $\SampleComp{\epsilon}{\delta_1}$
i.i.d examples, the learner outputs a hypothesis $h$ which with probability at
least $1 - \delta_1 \geq 1 - \delta_2$ has a true error of at most $\epsilon$.
This implies that for the $(\epsilon, \delta_2)$ combination of parameters, we
can bound the true error of $h$ by $\epsilon$ by using a smaller number of
i.i.d examples than $\SampleComp{\epsilon}{\delta_2}$. This again contradicts
the minimality of the sample complexity function. Hence we must have
$\SampleComp{\epsilon}{\delta_1} \geq \SampleComp{\epsilon}{\delta_2}$.

\section*{Exercise 3.2}

Given a sample $S$, we output a hypothesis $h_S$ with the property that
$\forall x \in S_x$,
\[
    h_S(x) = \left \{ \begin{array}{rl}
                            1, & \text{if $(x, 1) \in S$} \\
                            0, & \text{otherwise}
                      \end{array} \right .
\]
For any sample $S$, this hypothesis has an empirical loss of $0$. Note
that $h_S$ disagrees with the true labeling function $f$ in at most one point
$z \in \mathcal{X}$. It's true loss is therefore
$\Pr_{x \sim \mathcal{D}} \{ f(x) \neq h_S(x)\} = \Pr_{\mathcal{D}} \{z\} := p_z$.

The true loss of $h_S$ will be $0$ if $(z, 1) \in S$. Therefore the probability
of getting a ``bad'' sample is $\Pr_{S \sim \mathcal{D}^m}\{(z, 1) \notin S\}$.
Let $z^{*} \in \mathcal{X}$ be a point at which $(1 - p_z)^m$ is maximized. Since
$(1 - p_{z^{*}})^m \leq e^{- m p_{z^{*}}}$ and since we want the probability of
picking a bad sample to be at most $\delta$, we want $e^{- m p_{z^{*}}} < \delta$,
which gives us the sample size to be:
\begin{equation}
\label{3.2_samplecomp}
	m  > \frac{\log (1 / \delta)}{p_{z^{*}}}
\end{equation}

Depending on the value of the error bound $\epsilon$, there are two situations
to consider. If $\epsilon \geq p_{z^{*}}$, then even a sample of size one will
guarantee that the true error of $h_s$ is at most $\epsilon$. However if
$\epsilon < p_{z^{*}}$ then we can then use this in~(\ref{3.2_samplecomp})
to obtain:
\[
    m > \frac{\log (1 / \delta)}{\epsilon}.
\]
Thus the sample complexity is $\SampleComp{\epsilon}{\delta} =
\max \left \{1, \frac{\log (1 / \delta)}{\epsilon} \right \}$.

\section*{Exercise 3.3}

Here $\dom = \Rtwo$ and $\range = \{0, 1\}$. The hypothesis class $\hypclass$
is the set of concentric circles in $\Rtwo$ centered at the origin. Assuming
realizability, this implies that the true labeling function~$f = h_r$ for some
$r \in \Rpos$. Thus $f$ assigns the label $1$ to any point $(x, y) $ that is
within a distance of $r$ from the origin and $0$ otherwise.

Given any sample~$S$, let $q \in \Rpos$ be the minimum real number such that
all $(x, y) \in S_x$ with a label of $1$ are included in a circle centered at
the origin with radius~$q$. The output of the ERM procedure is $h_q$. The
empirical error of $h_q$ is zero, but it's true error is:
\[
	\Prtwo{(x, y) \sim \dist}{(x, y) \in C_r \setminus C_q}
\]
where $C_r$ and $C_q$ are concentric circles centered at the origin with
radius~$r$ and~$q$ respectively. Given an $\epsilon > 0$, let $t \in \Rpos$ be
such that $$\epsilon = \Prtwo{(x, y) \sim \dist}{(x, y) \in C_r \setminus
C_t}.$$ That is, we choose~$t$ so that the true error matches the probability
of picking anything inside the ring described by the circles $C_r$ and $C_t$.
Then the probability that we fail to choose any point in this ring in an i.i.d
sample of size~$m$ is $(1 - \epsilon)^m \leq e^{- \epsilon m}$. This is the
probability that we are handed a ``bad'' sample. Upper bounding this by
$\delta$, we obtain that $m > \log(1 / \delta) / \epsilon$.

Now a sample of size at least $\log(1 / \delta) / \epsilon$ has with
probability at least $1 - \delta$ a point from $C_r \setminus C_t$, and hence
the true error of the resulting ERM hypothesis is at most $\epsilon$. Hence the
sample complexity is upper bounded by $\ceiling{\log(1 / \delta) / \epsilon}$.

\section*{Exercise 3.4}

In this example, $\dom = \{0, 1\}^d$, $\range = \{0, 1\}$ and the hypothesis
class~$\hypclass$ is the set of all conjunctions over $d$ Boolean variables.
Since there are $\sum_{i = 0}^{d} {d \choose i} 2^i = 3^{d}$ Boolean
conjunctions over $d$ Boolean variables, the hypothesis class is finite. Hence
the sample complexity is
\begin{align*}
	\SampleComp{\epsilon}{\delta} & = \ceiling{\frac{\log (\hypclass / \delta)}{\epsilon}} \\
	& = \ceiling{\frac{d \cdot \log 3 + \log (1 / \delta)}{\epsilon}}
\end{align*}

To prove that the class $\hypclass$ is PAC-learnable, it suffices to exhibit a
polynomial-time algorithm that implements the ERM rule. The algorithm outlined
in Figure~\ref{fig:pac_boolean_conjunctions} starts with the formula $x_1 \land
\bar{x}_1 \land \cdots \land x_d \land \bar{x}_d$. It runs through the positive
examples in the sample $S$ and for each such example, it adjusts the formula so
that it satisfies the assignment given in the example. At the end of this
procedure, the modified formula satisfies all positive examples of $S$. The
time taken is $O(d \cdot |S|)$.

What may not be immediately apparent is that the formula returned by the
algorithm satisfies all negative examples too. This is clear when the sample
$S$ has \emph{no} positive examples to begin with as every assignment to $x_1
\land \bar{x}_1 \land \cdots \land x_d \land \bar{x}_d$ results in a $0$. The
point is that if there is even \emph{one} positive example, for each $1 \leq i
\leq d$, the algorithm eliminates either $x_i$ or $\bar{x}_i$ depending on the
assignment. That is, it eliminates half of the literals on seeing that one
example and the modified formula $f$ contains the literals of the true labeling
function along with possibly others.  Now the literals of the true labeling
function produce a $0$ on all negative examples and so does $f$. Hence the
sampling error of the function returned by the algorithm is $0$.
\begin{figure}
    \begin{algorithmic}[0] % The number tells where the line numbering should start
     \Procedure{PACBoolean}{$S$}
         \Comment{$S$ is the sample set with elements
         $\angular{(a_1, \ldots, a_d), b}$,
         where $(a_1, \ldots, a_d) \in \{0, 1\}^d$ and $b \in \{0, 1\}$}
            \State $f \gets x_1 \land \bar{x}_1 \land \cdots
            \land x_d \land \bar{x}_d$
            \For{each $\angular{(a_1, \ldots, a_d), b} \in S$ with $b = 1$}
            	\For{$j$ in $[1, \ldots, d]$}
            		\If{$a_j = 1$}
            			\State Delete $\bar{x}_j$ from $f$, if it exists in the formula
            		\Else
            			\State Delete $x_j$ from $f$, if it exists in the formula
            		\EndIf
             	\EndFor
            \EndFor
            \State \textbf{return} $f$
        \EndProcedure
    \end{algorithmic}
\caption{Learning Boolean conjunctions}
\label{fig:pac_boolean_conjunctions}
\end{figure}


\section*{Exercise 3.5}

The first thing to verify is that $\bar{\dist}_m$ is a distribution. This is
easy since for all $x \in \dom$, $\bar{\dist}_m(x) \geq 0$ and
\begin{align*}
	\int_{x \in \dom} \bar{\dist}_m (x) \dx x & = \frac{1}{m} \sum_{i = 1}^m
	\int_{x \in \dom} \dist_i (x) \dx x \\
	& = \frac{1}{m} \sum_{i = 1}^m 1 \\
	& = 1.
\end{align*}

Fix an accuracy parameter $\epsilon > 0$. As in the text, define the set of bad
hypotheses to be $\hypclass_{B} = \{h \in \hypclass \colon L_{\bar{\dist}_m, f}
(h) > \epsilon\}$ and let $\mathcal{M} = \{S \vert_x \colon \exists h \in
\hypclass_{B}, L_S(h) = 0\}$ be the set of misleading samples. Since we assume
realizability, any hypothesis~$h$ output by the ERM procedure has $L_S(h) =
0$. Thus the event $L_{\bar{\dist}_m, f} (h) > \epsilon$ and $L_S(h) = 0$
happens only when $S \vert_x \in \mathcal{M}$. Hence,
\begin{align*}
	\Prtwo{\forall i \colon x_i \sim \dist_i}{S \vert_x \in \mathcal{M}}
	& = \Prtwo{\forall i \colon x_i \sim \dist_i}{\bigcup_{h \in \hypclass_{B}}
			   \{S \vert_x \colon L_S(h) = 0\}} \\
    & \leq \sum_{h \in \hypclass_B}
    		\Prtwo{\forall i \colon x_i \sim \dist_i}{S \vert_x \colon L_S(h) = 0} \\
    & = \sum_{h \in \hypclass_B}
    		\prod_{i = 1}^{m} \Prtwo{x_i \sim \dist_i}{f(x_i) = h(x_i)} \\
    & =  \sum_{h \in \hypclass_B}
    		\prod_{i = 1}^{m} \left ( 1 - L_{\dist_i, f} (h) \right ) \\
    & \leq \sum_{h \in \hypclass_B} \left [
            \frac{1}{m} \sum_{i = 1}^{m} \left ( 1 - L_{\dist_i, f} (h) \right )
        \right ]^m \\
    & \leq \sum_{h \in \hypclass_B} \left [ 1 - L_{\bar{\dist}_{m}, f} (h)\right ]^m
\end{align*}

The second-last inequality follows from the fact that the arithmetic mean of a set of numbers
is at most their geometric mean. The quantity
$\sum_{h \in \hypclass_B} [ 1 - L_{\bar{\dist}_{m}, f} (h)]^m$
is at most $| \hypclass | \cdot (1 - \epsilon)^m $ which is at most
$| \hypclass | \cdot e^{- \epsilon m}$.

\section*{Exercise 3.6}

Agnostic PAC-learnability implies PAC-learnability. Let $\hypclass$ be a set of
functions from $\dom$ to $\{0, 1\}$ which is agnostic PAC-learnable wrt $\dom
\times \{0, 1\}$ and the 0-1 loss function with sample complexity
$m_{\mathcal{\hypclass}}$.  Let $f$ be a labeling function and let
$\dist_{\dom}$ be a distribution over $\dom$ for which the realizability
assumption holds, that is, there exists $h \in \hypclass$ such that
$L_{\dist_{\dom}, f } (h) = 0$.

Define a distribution $\dist$ over $\dom \times \{0, 1\}$ as follows: for all
$x \in \dom$, $\dist( (x, f(x)) ) = \dist_{\dom}(x)$ and $\dist( (x, 1 - f(x)))
= 0$. Fix $\epsilon, \delta > 0$. Since $\hypclass$ is agnostic PAC-learnable,
there exists a learner $A$ which given a sample $S$ of $m \geq
\SampleComp{\epsilon}{\delta}$ iid examples generated by $\dist$ returns a
hypothesis $h_S$ such that
\[
    \Prtwo{S \sim \dist^m}{L_{\dist} (h_S) \leq
        \min_{h' \in \hypclass} L_{\dist} (h') + \epsilon} > 1 - \delta.
\]

Note that for any $h' \in \hypclass$, we may write the loss $L_{\dist} (h')$ as follows.
\begin{align*}
    L_{\dist} (h') & = \Prtwo{(x, y) \in \dist}{ h'(x) \neq y} \\
                   & = \Prtwo{(x, y) \in \dist}{ h'(x) \neq f(x)} \\
                   & = L_{\dist_{\dom}, f} (h').
\end{align*}
The second equality above follows since the only points $(x, y) \in \dom \times \{0, 1\}$
for which $\dist$ places a non-zero probability mass are those for which $y = f(x)$. Since
we assume realizability, $ \min_{h' \in \hypclass} L_{\dist_{\dom}, f} (h') = 0$. Hence
the hypothesis $h_S$ returned by the learner $A$ satisfies:
\[
    \Prtwo{S|_{\dom} \sim \dist_{\dom}^m}{L_{\dist_{\dom}, f} (h_S) \leq \epsilon}
        > 1 - \delta,
\]
which is the condition for successful PAC-learnability.

\section*{Exercise 3.7} Let us fix some notation. We assume that $X$ and $Y$ are
random variables defined over the domains $\dom$ and $\{0, 1\}$, respectively.
Let $\dist_{X, Y}$ be a distribution over $\dom \times \{0, 1\}$;
let $\dist_{Y \mid X}$, the conditional distribution of $Y$ given $X$;
let $\dist_{X}$ be the marginal distribution of $X$ over $\dom$; and, finally, let
$\eta(x) = \Prtwo{\dist_{Y \mid X}}{Y = 1 \mid X = x}$.

Using this notation, the Bayes optimal classifier $f_{\dist}$ may be written as:
\[
    f_{\dist} (x) = \left \{ \begin{array}{ll}
                                1 & \text{if } \eta(x) \geq 1/2 \\
                                0 & \text{otherwise}
                         \end{array} \right .
\]
Given any classifier $g \colon \dom \rightarrow \{0, 1\}$, the risk of this
classifier is
\begin{equation}
\label{ex3.7:loss}
    L_{\dist} (g) = \Prtwo{\dist_{X, Y}}{g(X) \neq Y} = \int_{x \in \dom}
 \Prtwo{\dist_{Y \mid X}}{g(x) \neq Y \mid X = x} \cdot \Prtwo{\dist_{X}}{X = x} \dx x,
\end{equation}
where the second equality follows from the Law of Total Probability.
We may write the first term of this intergrand as follows (where all probabilities
are with respect to the conditional distribution $\dist_{Y \mid X}$):
\begin{align*}
    \Prone{g(x) \neq Y \mid X = x} & =  1 - \Prone{g(x) = Y \mid X = x} \\
    & = 1 - \left [ \Prone{g(x) = 1, Y = 1 \mid X = x} + \Prone{g(x) = 0, Y = 0 \mid X = x} \right ] \\
    & = 1 - \left [ \ind_{g(x) = 1} \cdot \Prone{Y = 1 \mid X = x}  +
                  \ind_{g(x) = 0} \cdot \Prone{Y = 0 \mid X = x} \right ] \\
    & = 1 - \left [ \ind_{g(x) = 1} \cdot \eta (x)  +
                  \ind_{g(x) = 0} \cdot (1 - \eta (x)) \right ] \\
\end{align*}

Consider the difference $\Prone{g(x) \neq Y \mid X = x} - \Prone{f_{\dist}(x) \neq Y \mid X = x}$.
This may be written as:
\begin{multline*}
\left [ \ind_{f_{\dist} (x) = 1} \cdot \eta (x)  +
        \ind_{f_{\dist} (x) = 0} \cdot (1 - \eta (x)) \right ] -
\left [ \ind_{g(x) = 1} \cdot \eta (x)  +
        \ind_{g(x) = 0} \cdot (1 - \eta (x)) \right ]  \\
  = \left ( \ind_{f_{\dist} (x) = 1} - \ind_{g(x) = 1} \right ) \cdot \eta(x) +
  \left ( \ind_{f_{\dist} (x) = 0} - \ind_{g(x) = 0} \right ) \cdot (1 - \eta(x)).
\end{multline*}
Since $\ind_{f_{\dist} (x) = 0} = 1 - \ind_{f_{\dist} (x) = 1}$ and
$\ind_{g (x) = 0} = 1 - \ind_{g(x) = 1}$, this last expression may be written as:
\[\left ( \ind_{f_{\dist} (x) = 1} - \ind_{g(x) = 1} \right ) \cdot \eta(x) +
  \left ( \ind_{g (x) = 1} - \ind_{f_{\dist} (x) = 1} \right ) \cdot (1 - \eta(x)).
\]
Rearranging terms allows us to write this as:
\begin{equation}
%2 \cdot \left ( \ind_{f_{\dist} (x) = 1} - \ind_{g(x) = 1} \right ) \cdot \eta(x) +
%  \left ( \ind_{g (x) = 1} - \ind_{f_{\dist} (x) = 1} \right ).
 (2 \eta (x) - 1) \cdot (\ind_{f_{\dist} (x) = 1} - \ind_{g(x) = 1}).
\end{equation}
We claim that this last expression is always non-negative. If $f_{\dist} (x) = 0$ then
$\eta (x) < 1/2$ and the above expression is non-negative. If $f_{\dist} (x) = 1$ then
$\eta (x) \geq 1/2$ and, in this case too, the expression is non-negative. The result
follows by plugging in the difference
$\Prone{g(x) \neq Y \mid X = x} - \Prone{f_{\dist}(x) \neq Y \mid X = x}$
in the integral in~(\ref{ex3.7:loss}).





