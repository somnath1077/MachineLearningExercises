\chapter{Non-Uniform Learnability}

In the (agnostic) PAC learning setting, the sample size depended only on the
accuracy parameter~$\epsilon$ and the confidence parameter~$\delta$. It was ``uniform''
w.r.t the hypothesis class and the underlying distribution. In non-uniform
learnability, the sample size is allowed to depend on the hypothesis class.
In particular, when a learning algorithm is competing against a specific hypothesis
in the hypothesis class, then it is allowed to have a training sample size that depends
on that hypothesis (and also on $\epsilon, \delta$).

\begin{definition}
Let $\hypclass$ be a set of binary functions over a domain $\dom$.  The class
$\hypclass$ is \emph{non-uniformly learnable} if there exist a learning
algorithm $\algo$ and a function $\nusamplecomp \colon (0, 1) \times (0, 1)
\times \hypclass \rightarrow \Nat$ such that for all $\epsilon, \delta \in (0,
1)$, for all $h \in \hypclass$ and all distributions $\dist$ over
$\dom \times \{0, 1\}$, the following holds for samples $S$
with $m \geq \nusamplecomp(\epsilon, \delta, h)$ examples:
\[
    \Prtwo{S \sim \dist^m}{L_{\dist}(\algo (S)) \leq L_{\dist} (h) + \epsilon}
        \geq 1 - \delta.
\]
\end{definition}

It should be clear that non-uniform learnability is a relaxation of agnostic
PAC-learnability in that if a hypothesis class is agnostic PAC-learnable then it
is also non-uniformly learnable. In fact, the notion of non-uniform learnability
is a strict generalization of the notion of agnostic PAC-learnability. This can be
shown by considering the set of all thresholded polynomials. The VC-dimension of
this class is infinite and hence it is not agnostic PAC-learnable, but
it is non-uniformly learnable as the set of thresholded polynomials of any
fixed degree~$d$ has VC-dimension~$d + 1$. The non-uniform learnability of all
thresholded polynomials follows from the following characterization:

\begin{theorem} \label{thm:non_uniform}
A hypothesis class $\hypclass$ of binary classifiers is non-uniformly learnable iff it
is the countable union of agnostically PAC-learnable hypothesis classes.
\end{theorem}

On the other hand, the notion of non-uniform learnability is restricted enough
that it does not allow all hypothesis classes to be learnable. This follows
from Lemma~\ref{lemma:non_uniform} and the characterization of non-uniform
learnable classes in Theorem~\ref{thm:non_uniform}.

\begin{lemma}\label{lemma:non_uniform}
For every infinite set $\dom$, the set of all binary valued functions on $\dom$ is
not expressible as $\bigcup_{n = 1}^{\infty} H_n$, where each $H_n$ has finite VC-dimension.
\end{lemma}

The main topic is in proving Theorem~\ref{thm:non_uniform}. We do this in two steps.
\begin{lemma}\label{lemma:if_side}
If $\hypclass$ is non-uniformly learnable then there exist classes $H_n$, $n \in \Nat$, each
of finite VC-dimension, such that $\hypclass = \bigcup_{n = 1}^{\infty} H_n$.
\end{lemma}
\begin{proof}
Let $\hypclass$ be as stated and for each $n \in \Nat$, define
\[
    H_n := \{h \in \hypclass \colon \nusamplecomp (1/8, 1/7, h) \leq n\}.
\]
Then clearly $H = \bigcup_{n = 1}^{\infty} H_n$. By definition, each class $H_n$
can be learned to an accuracy of $1/8$ with a confidence of $1/7$ using at most $n$
examples. We claim that $\vcdim (H_n) \leq 2n$ for all
$n \in \Nat$. Suppose not. Then there exist $n \in \Nat$ and a set $A \subset \dom$ of size
$2n + 1$ that is shattered by $H_n$. This means that every binary function on $A$ admits
an extension in $H_n$ to a function on $\dom$. By the No-Free-Lunch Theorem, in order
to be able to learn $H_n$ to an accuracy of $1/8$ with a confidence of $1/7$, we need
at least $n + 1$ examples. This contradiction shows that there cannot be any set
of size $2n + 1$ that is shattered by $H_n$. Hence $\vcdim (H_n) \leq 2n$.
\qedhere
\end{proof}


To prove the other direction, we need to introduce the notion of the \emph{Structural
Risk Minimization (SRM)} paradigm. Let $\hypclass = \bigcup_{n = 1}^{\infty} \hypclass_n$
be a hypothesis class that is the union of a countably many uniformly convergent
classes $\hypclass_n$ with sample complexity function
$m_n (\epsilon, \delta)$. Define $\epsilon_n \colon \Nat \times (0, 1) \rightarrow (0, 1)$
by
\[
    \epsilon_n (m, \delta) :=
        \min \{\epsilon \in (0, 1) \colon m_n (\epsilon, \delta) \leq m\}.
\]
Thus $\epsilon_n (m, \delta)$ is just the best possible accuracy obtainable
when learning functions from the class $\hypclass_n$ with a sample size of
at most $m$ and with confidence $\delta$.

Since a given $h \in \hypclass$ may belong to an infinite number of $\hypclass_n$,
define $n(h) := \min \{n \in \Nat \colon h \in \hypclass_n \}$. The other
component of SRM is a weight function $w \colon \Nat \rightarrow [0, 1]$ such
that $\sum_{n = 1}^{\infty} w(n) \leq 1$. The weight function $w(n)$ represents
the importance that the learner attributes to each hypothesis class $\hypclass_n$.

\begin{theorem}
\label{thm:SRM1}
Let $\hypclass$ be a hypothesis class that can be written as $\hypclass =
\bigcup_{n = 1}^{\infty} \hypclass_n$, where each class $\hypclass_n$ is
uniformly convergent with sample complexity $m_n(\epsilon, \delta)$. Let $w
\colon \Nat \rightarrow [0, 1]$ be a weight function such that  $\sum_{n =
1}^{\infty} w(n) \leq 1$. Then for every $\delta \in (0, 1)$, for every
distribution $\dist$, with probability of at least $1 - \delta$ over the choice
of samples $S \sim \dist^m$, the following holds for all $n \in \Nat$ and all
$h \in \hypclass_n$:
\[
    |L_{\dist} (h) - L_S (h)| \leq \epsilon_n (m, w(n) \cdot \delta).
\]
\end{theorem}
The main claim of this theorem can be restated as either
one of these statements:
\begin{itemize}
    \item for all $h \in \hypclass$:
        $L_{\dist} (h) \leq L_S(h) + \epsilon_n (m, w(n) \cdot \delta)$
    \item for all $h \in \hypclass$:
    $L_{S} (h) \leq L_{\dist}(h) + \epsilon_n (m, w(n) \cdot \delta)$
\end{itemize}

Given a training sample $S$ and a confidence parameter $\delta$,
the SRM paradigm is to select a hypothesis $h \in \hypclass$ that minimizes
\[
    L_S(h) + \epsilon_{n(h)} (m, w( n(h) ) \cdot \delta).
\]
\begin{theorem}
\label{thm:SRM2}
Let $\hypclass$ be a hypothesis class that can be written as $\hypclass =
\bigcup_{n = 1}^{\infty} \hypclass_n$, where each class $\hypclass_n$ is
uniformly convergent with sample complexity $m_n(\epsilon, \delta)$. Let $w
\colon \Nat \rightarrow [0, 1]$ be a weight function such that  $\sum_{n =
1}^{\infty} w(n) \leq 1$. Then $\hypclass$ is non-uniformly learnable
using the SRM rule with rate:
\[
    \nusamplecomp (\epsilon, \delta, h) \leq m_{n(h)} (\epsilon / 2, w( n(h) ) \cdot \delta).
\]
\end{theorem}
\begin{proof}
Let $\algo$ be the SRM algorithm. Given $\epsilon$, $\delta$ and $h \in \hypclass$,
choose $m \geq m_{n(h)} (\epsilon / 2, w( n (h) ) \cdot \delta)$. By Theorem~\ref{thm:SRM1},
for every distribution $\dist$, with probability at least  $1 - \delta$ over the choice
of samples $S \sim \dist^m$ for all $h' \in \hypclass$:
\[
    L_{\dist} (h') \leq L_S(h') + \epsilon_{n(h')} (m, w( n(h') ) \cdot \delta).
\]
Since $\algo$ uses the SRM rule:
\begin{align*}
    L_{\dist} (A(S))
        &  \leq \min_{h' \in \hypclass} \{
            L_S(h') + \epsilon_{n(h')} (m, w( n(h') ) \cdot \delta) \} \\
        & \leq L_S(h) +  \epsilon_{n(h)} (m, w( n(h) ) \cdot \delta). \\
\end{align*}
By Theorem~\ref{thm:SRM1}, with probability at least  $1 - \delta$ over the choice of samples
$S \sim \dist^m$:
\begin{align*}
    L_S(h) +  \epsilon_{n(h)} (m, w( n(h) ) \cdot \delta)
        & \leq L_{\dist}(h) +  2 \cdot \epsilon_{n(h)} (m, w( n(h) ) \cdot \delta) \\
        & \leq L_{\dist}(h) + 2 \cdot \epsilon / 2 \\
        & \leq L_{\dist}(h) + \epsilon.
\end{align*}
\end{proof}

An immediate implication is that every countably infinite hypothesis class for
binary classification is non-uniformly learnable using a suitably defined weight
function. For if, $\hypclass = \bigcup_{n = 1}^{\infty} \{ h_n \}$ then, by
Corollary~4.6, each singleton class $\{ h_n \}$ has the uniform convergence property with
sample complexity $m (\epsilon, \delta) = \frac{\log (2 / \delta)}{\epsilon^2}$.


\section*{Ex 7.1}

Let $l$ be the maximum description length of a hypothesis in $\hypclass$. Since
the descriptions are over the alphabet $\{0, 1\}$, the maximum number of hypotheses
that can possibly be represented is:
\[
    2^{l} + 2^{l - 1} + \cdots + 2^1 + 2^0 = 2^{l + 1} - 1.
\]
Hence $|\hypclass| \leq 2^{l + 1} - 1 < 2^{l + 1}$. Since
$\vcdim (\hypclass) \leq \log_2 |\hypclass|$, we have that:
$\vcdim (\hypclass) < l + 1$.

If, in addition, the descriptions are prefix-free, then by Kraft's
inequality:
\[
    \frac{|\hypclass|}{2^l} \leq \sum_{h \in \hypclass} \frac{1}{2^{|h|}} \leq 1,
\]
which yields $|\hypclass| \leq 2^l$, and hence $\vcdim (\hypclass) \leq l$.

\section*{Ex 7.2}

It is sufficient to prove that there exists no weighting function
$w \colon \Nat \rightarrow [0, 1]$ that is not indentically zero such that
\begin{enumerate}
    \item $\sum_{n = 1}^{\infty} w(n) \leq 1$
    \item $w$ is non-decreasing: for all $i < j$, $w(i) \leq w(j)$.
\end{enumerate}
Since, by hypothesis, $w$ is not identically zero, there exists an index $n_0$
such that $w_{n_0} > 0$. Since $w$ is non-decreasing, this implies that for
all $n \geq n_0$, $w(n) \geq w(n_0)$. Hence
$\sum_{i = 0}^{k} w( n_0 + i ) \geq k \cdot w( n_0 )$, which implies that
the sum $\sum_{i = 1}^{\infty} w(i)$ diverges to infinity, contradicting
requirement~$(1)$.

\section*{Ex 7.3}

\subsection*{Ex 7.3.1}

Let $\hypclass = \bigcup_{n \in \Nat} \hypclass_n$, where $\hypclass_n$ is finite
for all $n \in \Nat$. For $h \in \hypclass$, define
\[
    w(h) = \frac{1}{|H_{ n(h) }| \cdot 2^{ n(h) }}.
\]
Then
\begin{align*}
\sum_{h \in \hypclass} w(h) & = \sum_{h \in \hypclass} \frac{1}{|H_{ n(h) }| \cdot 2^{ n(h) }} \\
    & \leq \sum_{j = 1}^{\infty} \frac{|H_j|}{|H_j| \cdot 2^j} \\
    & \leq 1.
\end{align*}


\subsection*{Ex 7.3.2}

Let $\hypclass = \bigcup_{n \in \Nat} \hypclass_n$, where $\hypclass_n$ is countably infinite.
In this case, every $h \in \hypclass$ can be uniquely identified by a pair of natural numbers
$(i, j)$, where $i = n(h)$ and $j = \id (h)$ is the index of $h$ in $\hypclass_{ n(h) }$. Now define $w(h)$
as $\frac{1}{2^{n(h) + \id(h)}}$. With this definition, a countably infinite number of $h \in \hypclass$
may have the same first component $n(h)$, but since each hypothesis
has a distinct index~$j$ in $\hypclass_{n(h)}$, we ensure that the weights associated
with the class $\hypclass_{n(h)}$ are decreasing. Indeed,
\begin{align*}
    \sum_{h \in \hypclass} w(h) & = \sum_{h \in \hypclass} \frac{1}{2^{n(h) + \id(h)}} \\
        & \leq \sum_{n = 1}^{\infty} \sum_{j = 1}^{\infty} \frac{1}{2^{n + j}} \\
        & \leq \sum_{n = 1}^{\infty} \frac{1}{2^n} \sum_{j = 1}^{\infty} \frac{1}{2^j} \\
        & \leq \sum_{n = 1}^{\infty} \frac{1}{2^n} \cdot 1 \\
        & \leq 1.
\end{align*}

\section*{Ex 7.5}

A No-Free-Lunch result for non-uniform learnability.
\begin{theorem}
\label{thm:NFL_NUL}
Let $\hypclass$ be a class that shatters an infinite set. Then for every
sequence of classes $\{ \hypclass_n \}_{n \in \Nat}$ such that
$\hypclass = \bigcup_{n \in \Nat} \hypclass_n$, there exists some $n$
for which $\vcdim (\hypclass_n) = \infty$.
\begin{proof}
Let $\hypclass$ be a hypothesis class that shatters a countably infinite
set $K = \{k_i\}_{i = \Nat}$ and suppose that $\hypclass = \bigcup_{n \in \Nat} \hypclass_n$,
where every $\hypclass_n$ has finite VC-dimension. For $n \in \Nat$,
define $K_n \subseteq K$ such that $|K_n| = \vcdim (\hypclass_n) + 1$ and
$K_n \cap K_m = \emptyset$ for $n \neq m$. One possible definition that
satisfies these conditions is the following:
\[
    K_n = \{k_{r + 1}, \ldots, k_{r + \vcdim(\hypclass_n) + 1}\}
\]
where $r = \sum_{j = 1}^{n - 1} \vcdim (\hypclass_j)$. Clearly, $K_1, \ldots, K_n, \ldots$
are pairwise disjoint and $K = \bigcup_{n \in \Nat} K_n$.

Since $|K_n| > \vcdim (\hypclass_n)$, there exists a function $f_n \colon K_n \rightarrow \{0, 1\}$
that does not agree with any function in $\hypclass_n$ on $K_n$. Define
$f \colon K \rightarrow \{0, 1\}$ as follows: for $k \in K_n$, $f(k) = f_n(k)$. Since
$\hypclass$ shatters $K$, we must have $f \in \hypclass$. By the definition of $f$,
for all $n$, $f$ differs from every function in $\hypclass_n$ on the set $K_n$. Thus
$f \notin \bigcup_{n \in \Nat} \hypclass_n$, a contradiction to the assumption
that $\hypclass = \bigcup_{n \in \Nat} \hypclass_n$, where each $\hypclass_n$
has finite VC-dimension. This shows that if $\hypclass$ shatters a countably
infinite set, then it does not admit such a represtation. Since any
non-uniformly learnable class does in fact admit such a representation, it
follows that $\hypclass$ is not non-uniformly learnable.
\end{proof}
\end{theorem}
