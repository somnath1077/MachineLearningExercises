\chapter{The VC-Dimension}

\section*{Notes on Chapter 6}

We know that finite hypothesis classes are agnostic PAC learnable (and hence
PAC learnable).  What about infinite hypothesis classes? The first example is 
that of an infinite hypothesis class that is PAC learnable. 

\begin{example}[Threshold Functions]
Let $\dom = [0, 1]$ and $\range = \{0, 1\}$. For $r \in [0, 1]$, define 
$h_r \colon \dom \rightarrow \range $ as:
\[
    h_r(x) = \left \{ \begin{array}{ll} 
                        0 & \text{if } x \leq r \\
                        1 & \text{if } x > r
                      \end{array}\right .
\]
Let $\hypclass_{\text{thr}}$ be the set of all threshold functions $h_r$ 
for $r \in [0, 1]$. Since $\hypclass_{\text{thr}}$ is not finite, it is 
not immediately obvious whether it is PAC learnable (in the realizable case). 

Fix $\epsilon, \delta \in (0, 1)$. Let $f = h_s$ be the true labeling function
where $s \in [0, 1]$ and let $\dist$ be the underlying distribution over the
domain $[0, 1]$. Let $s_0 \in [0, s)$ and $s_1 \in [s, 1]$ be numbers such that  
\[
    \dist \left \{ x \in [s_0, s) \right \} = \epsilon =  
    \dist \left \{ x \in [s, s_1] \right \}
\]
If $\dist \left \{ [0, s) \right \} < \epsilon$, then set $s_0 = 0$; similarly, 
if $\dist \left \{ [s, 1] \right \} < \epsilon$, set $s_1 = 1$. Since $\dist$ 
is a distribution, it must place a probability mass of $\epsilon$ either to 
the left or to the right of $s$. 

Given a sample $S$, let $t_0 = \max \{t \colon (t, 0) \in S\}$ and $t_1 = \min
\{t \colon (t, 1) \in S\}$. The ERM algorithm outputs $h_p$, where $p \in (t_0,
t_1)$.  In particular, if the sample presented to the ERM algorithm is such
that $s_0 \leq t_0$ and $t_1 \leq s_1$, then hypothesis $h_p$ returned by the ERM
algorithm will incur a loss of $L_{\dist}(h_p) \leq \epsilon$.

Thus the probability that the hypothesis $\ERM(S)$ output by the ERM algorithm 
has a loss greater than $\epsilon$ on a sample $S$ of size $m$ is:
\begin{align*}
    \Prtwo{S \sim \dist^m}{L_{\dist} (\ERM(S)) > \epsilon} 
     & = \Prtwo{S \sim \dist^m}{S \colon t_0 < s_0 \vee s_1 < t_1} \\
     & \leq  \Prtwo{S \sim \dist^m}{S \colon S|_x \cap [s_0, s) = \emptyset} + 
             \Prtwo{S \sim \dist^m}{S \colon S|_x \cap [s, s_1] = \emptyset} \\      
     & \leq 2 \cdot (1 - \epsilon)^m \\
     & \leq 2 \cdot e^{- \epsilon m}
\end{align*}
Setting the last expression to be at most $\delta$, we obtain that 
$m > \frac{1}{\epsilon} \cdot \log \frac{2}{\delta}$. Hence if we have samples 
of size at least $\frac{1}{\epsilon} \cdot \log \frac{2}{\delta}$, 
\[
    \Prtwo{S \sim \dist^m}{L_{\dist} (\ERM(S)) \leq \epsilon} \geq 1 - \delta, 
\]
which is the condition for PAC learnability.
\end{example}

The second example shows that there are infinite hypothesis classes that are not
PAC learnable at least by using an ERM strategy.

\begin{example}[Identity Function for Finite Sets] 
Let $\dom = \Rone$ and $\range = \{0, 1\}$. Given a set $A \subseteq \dom$,
define $h_A$ as follows:
\[
    h_A = \left \{ \begin{array}{ll} 
                        1 & \text{if } x \in A \\
                        0 & \text{otherwise}
                   \end{array}\right .
\]
Let $\hypclass_{\text{finite}}$ be the set of all such functions $h_A$ for \emph{finite} 
subsets $A$ of $\Rone$ along with the function $h_{1}$ which maps every point in $\Rone$ 
to $1$. We claim that $\hypclass_{\text{finite}}$ is not PAC learnable by an ERM algorithm. 

Consider the case when the true labeling function $f = h_1$, the all-ones
function on $\Rone$ and $\dist$ is the uniform distribution on $[0, 1]$. Since
$f \in \hypclass_{\text{finite}}$, we are assuming that the hypothesis class is
realizable. Fix any sample size $m$. A sample $S$ in this case looks like
$\{(x_1, 1), \ldots, (x_m, 1)\}$ and an obvious ERM strategy is to output $h_A$
for $A = \{x_1, \ldots, x_m\}$. Clearly $L_S (h_A) = 0$ but $L_{\dist} (h_A) =
1$.  
\end{example} 

The previous examples show that the size of the hypothesis class does not characterize
whether it is learnable. This characterization is provided by the so-called VC-dimension.

\section*{Exercise 6.1}

Let $\hypclass$ be a set of functions from $\dom$ to $\{0, 1\}$ and
let $\hypclass' \subseteq \hypclass$. Assume that $\vcdim (\hypclass') > \vcdim
(\hypclass)$.  Then there exists a set $C \subseteq \dom$ that is shattered by
$\hypclass'$ but not by $\hypclass$.  This implies that for all $g \colon C
\rightarrow \{0, 1\}$ there exists $h' \in \hypclass'$ such that $g(x) = h'(x)$
for all $x \in C$. Since $h' \in \hypclass$, this implies that $\hypclass$
shatters $C$, a contradiction.

\section*{Exercise 6.2}

In this exercise, $\dom$ is finite and $k \leq |\dom| =: n$. 

\subsection*{6.2.1} 

We claim that 
\[\vcdim(\hypclass_{= k}) 
= \left \{ \begin{array}{ll} 
                k     & \text{if } k \leq \floor{n / 2} \\
                n - k & \text{if } k > \floor{n / 2} 
           \end{array} \right .
\]

Suppose that $k \leq \floor{n / 2}$ and consider a subset $C \subset \dom$ 
of size $k + 1$. Then the all-one function on $C$ cannot be 
extended to a function in $\hypclass_{= k}$ as it maps $k + 1$ elements
of $\dom$ to $1$. Hence $\vcdim(\hypclass_{= k}) \leq k$. If $|C| = k$ 
and $g \colon C \rightarrow \{0, 1\}$ that maps $k'$ elements of $C$ to $1$,
we can extend $g$ to a function on $\dom$ that maps exactly $k$ elements of 
$\dom$ to $1$.  This shows that $\vcdim(\hypclass_{= k}) \geq k$. Hence 
$\vcdim{\hypclass_{= k}} = k$.

Now consider the case $k > \floor{n / 2}$. If $C$ is subset of size
$n - k + 1$, then the all-zero function on $C$ cannot be extended 
to a function in $\hypclass_{= k}$. This happens because there are only 
$n - (n - k + 1) < k$ elements in $\dom \setminus C$. Hence 
$\vcdim(\hypclass_{= k}) \leq n - k$. If $|C| = n - k$ and 
$g \colon C \rightarrow \{0, 1\}$ that assigns $1$ to $k'$ 
elements of $C$, then we can extend $g$ to a function in 
$\hypclass_{= k}$ as we have at least $k - k'$ elements in 
$\dom \setminus C$ which we can map to $1$. This shows that 
$\vcdim(\hypclass_{= k}) \geq n - k$. Hence 
$\vcdim(\hypclass_{= k}) = n - k$.

\subsection*{6.2.2}

First observe that if $k \geq \floor{n / 2}$, then $\hypclass_{\leq k}$ includes 
all possible functions from $\dom$ to $\{0, 1\}$. This is because any function 
$g \colon \dom \rightarrow \{0, 1\}$ maps at most half the elements of 
$\dom$ to either $0$ or $1$ and hence is in $\hypclass_{\leq k}$. 
Hence in this case every subset of $\dom$ is shattered by $\hypclass_{\leq k}$
and $\vcdim (\hypclass_{\leq k}) = n$.

If $k < \floor{n / 2}$, then we claim that  $\vcdim (\hypclass_{\leq k}) = 2k + 1$.
Let $C \subset \dom$ of size $2k + 1$ and consider a function 
$g \colon C \rightarrow \{0, 1\}$. Such a function maps at most $k$ elements 
to either $0$ or $1$. Suppose that it maps at most $k$ elements to $1$. Extend
$g$ to a function on $\dom$ by mapping all elements of $\dom \setminus C$ to $0$. 
This extension is a function on $\dom$ that maps at most $k$ elements to $1$
and hence is an element of $\hypclass_{\leq k}$. The reasoning is similar had $g$ 
mapped at most $k$ elements to $0$. This show that 
$\vcdim(\hypclass_{\leq k}) \geq 2k + 1$. 

Now suppose that $C \subset \dom$ is of size $2k + 2$. Consider a map  
that assigns half the elements of $C$ to $0$ and the other half to $1$. 
This map cannot be extended to a function in $\hypclass_{\leq k}$. This 
proves that $\vcdim (\hypclass_{\leq k}) \leq 2k + 1$. Thus:
\[\vcdim(\hypclass_{\leq k}) 
= \left \{ \begin{array}{ll} 
                2k + 1  & \text{if } k < \floor{n / 2} \\
                n       & \text{if } k \geq \floor{n / 2} 
           \end{array} \right .
\]

\section*{Exercise 6.3}

Since $|\parity{n}| = 2^n$, using the upper bound on the VC-dimension, 
\[\vcdim (\parity{n}) \leq \log_2 |\parity{n}| = n.\]
We claim that $\vcdim (\parity{n}) = n$. Let $C = \{c_1, \ldots, c_n\} \subset
\dom$ be the set of standard basis vectors such that $c_i$ is the basis 
vector with a $1$ in the $i$th position and $0$'s
elsewhere. Let $(b_1, \ldots, b_n)$ be a function from $C$ to $\{0, 1\}$.
Construct an index set $I \subseteq \{1, \ldots, n\}$ as follows: 
start with $I \leftarrow \emptyset$; for $1 \leq i \leq n$, if $b_i = 1$ 
then $I \leftarrow I \cup \{i\}$. 

We claim that $h_I(c_j) = b_j$ for all $1 \leq j \leq n$. For if $b_j = 0$, 
then $j \notin I$ and $\sum_{i \in I} c_{j i} = 0 \Mod{2}$; if $b_j = 1$,
then $j \in I$ and $\sum_{i \in I} c_{j i} = 1 \Mod{2}$, proving the claim. 
This shows that every function from $C$ to $\{0, 1\}$ can be extended to a function
in $\parity{n}$. Hence $\vcdim (\parity{n}) \geq n$ and together with the upper 
bound for the VC-dimension, this implies that $\vcdim (\parity{n}) = n$.

\section*{Exercise 6.5}

Let $\rect{d}$ be the set of axis-aligned rectangles in $\Rone^{d}$. A function
in $\rect{d}$ is defined via $2d$ parameters $(a_1^1, a_2^1, a_1^2, a_2^2,
\ldots, a_1^d, a_2^d)$, where for $1 \leq i \leq d$, $a_1^i \leq a_2^i$ are the
boundaries of the rectangle in dimension~$i$. 

We claim that $\rect{d} = 2d$. Consider a set of $2d$ points that correspond to 
the centres of the faces of an axis-aligned rectangle in $\Rone^d$. For example,
if the faces of the rectangle are defined by the equations:
$x_i  = a_1^i$ and  $x_i = a_2^i$ for $1 \leq i \leq d$,
then the centres of the face defined by $x_1 = a_1^1$ and $x_1 = a_2^i$ are 
$( a_1^1,  \frac{a_1^2 + a_2^2}{2},  \ldots, \frac{a_1^d + a_2^d}{2})$ 
and $( a_2^1,  \frac{a_1^2 + a_2^2}{2},  \ldots, \frac{a_1^d + a_2^d}{2})$. 
Similarly, there are two points for each of the remaining dimensions, with a 
total of $2d$ points. Call this set of points $C$. Such a set $C$ will be shattered
by $\rect{d}$. 

On the other hand, no subset $C'$ of $\Rone^d$ of size $2d + 1$ can be
shattered by $\rect{d}$. The reasoning is similar to that given in the book.
For each dimension~$i$ select a point~$c_{\text{min}}^i \in C'$ whose $i$th
co-ordinate is a minimum among all points in $C'$; also select
$c_{\text{max}}^i \in C'$ whose $i$th co-ordinate is a maximum. This procedure
yields $2d$ points and the rectangle that contains all these $2d$ points must
necessarily contain the $2d + 1$st point. Hence a function that maps these $2d$
points to $1$, and the $2d + 1$st point to $0$ cannot be extended in
$\rect{d}$, proving that the set cannot be shattered. 

\section*{Exercise 6.6}

\begin{enumerate}
    \item $|\conj{d}| \leq 3^d + 1$. One way of counting the number of boolean 
        conjunctions is to first select a set of indices from among 
        $\{1, \ldots, d\}$ and then from among these select either a positive 
        or the negative version of the variables. When we select no indices, 
        we obtain the all-positive hypothesis. The number of such conjunctions
        is:
        \[
                \sum_{i = 0}^{d} 2^i = 3^d.
        \] 
        This does not include the all-negative conjunction. Hence the \emph{total}
        number of conjunctions is $3^d + 1$, which is a tight upper  bound for 
        $|\conj{d}|$.

    \item $\vcdim (\conj{d}) \leq 3 \cdot \log_2 d$. This immediately follows 
        from the upper bound in the text.

    \item Let $\basisvec_1, \ldots, \basisvec_d$ be the standard basis vectors of
        $\{0, 1\}^d$. Let $(b_1, \ldots, b_d)$ be a mapping from this set of 
        basis vectors to $\{0, 1\}$. Note that there are only $2^d$ such functions 
        and we will show that each such function can be represented by a 
        boolean conjunction on $d$ variables. 
    
        Start with the conjunction: $f := x_1 \wedge \bar{x}_1 \wedge 
        \cdots \wedge x_{d} \wedge \bar{x}_d$; for $1 \leq i \leq d$, 
        if $b_i = 1$ then drop $\bar{x}_i$ and all $x_j$ for $j \neq i$ from $f$. 
        Note that after this step we have that: $f(\basisvec_j) = b_j$ for all $j \leq i$.
        Thus at step $d$, we end up with a formula $f$ that matches the function 
        $(b_1, \ldots, b_d)$.
        Thus the set of basis vectors is shattered by $\conj{d}$ and 
        hence $\vcdim (\conj{d}) \geq d$.

    \item $\vcdim (\conj{d}) \leq d$. Suppose that $C = \{c_1, \ldots, c_{d + 1}\}$
        is shattered by $\conj{d}$. This implies that every function from $C$ to 
        $\{0, 1\}$ can be extended to a function in $\conj{d}$. Consider the 
        $d + 1$ functions $g_1, \ldots, g_{d + 1}$, where $g_i$ maps $c_i$ to $0$ 
        and all $c_j$ to $1$ for $j \neq i$. Let $h_1, \ldots, h_{d + 1}$ be the 
        extensions of these functions in $\conj{d}$. Then for each $i \in 
        \{1, \ldots, d+ 1\}$, there exists a literal $l_i$ in $h_i$ such that 
        $l_i$ is false for $c_i$ but true for all $c_j$, $j \neq i$. Furthermore, 
        each $h_i$ has at most $d$ literals since none of these functions is  
        the all-zero function. By the Pigeon Hole Principle, there exists $i$, 
        such that $1 \leq i \leq d$ and $l_i$ and $l_{d + 1}$ use the same variable, 
        say $x_k$. 

        Since $h_i$ maps $c_i$ to $0$ and $c_{d + 1}$ to $1$ and $h_{d + 1}$ maps
        these the other way around, it cannot be that both $l_i$ and $l_{d + 1}$ 
        use $x_k$ in the same form. That is, either $l_i = x_k$ and $l_{d+1} = 
        \bar{x}_k$ or vice versa. Consider the effect of the literals on the 
        bit strings in $C \setminus \{c_i, c_{d+1}\}$. Both map each bit string to 
        $1$, an impossibility since $l_i$ and $l_{d + 1}$ will have the opposite 
        effect on each of these bit strings too. This contradition shows that the 
        assumption that there exist functions $h_1, \ldots, h_{d + 1}$ that extend 
        $g_1, \ldots, g_{d + 1}$ is incorrect. Thus $C$ is not shattered by $\conj{d}$
        and $\vcdim (\conj{d}) \leq d$. 

    \item $\vcdim (\mconj{d}) = d$. Since $\mconj{d} \subseteq \conj{d}$, we know 
        that $\vcdim (\mconj{d}) \leq d$. Now consider $C = \{c_1, \ldots, c_d\} 
        \subset \{0, 1\}^d$, where $c_i$ has ones in all locations except the $i$th. 
        Let $(b_1, \ldots, b_d)$ be any function from $C$ to $\{0, 1\}$.  
        Start with the conjunction: $f := x_1 \wedge 
        \cdots \wedge x_{d}$; for $1 \leq i \leq d$, 
        if $b_i = 1$ then drop $x_i$ from $f$. 
        Note that after this step we have that: $f(c_j) = b_j$ for all $j \leq i$.
        Thus at step $d$, we end up with a formula $f$ that matches the function 
        $(b_1, \ldots, b_d)$.
        Thus the set $C$ is shattered by $\mconj{d}$ and hence 
        $\vcdim (\mconj{d}) = d$.
\end{enumerate}

\section*{Exercise 6.7}

\subsection*{6.7.1} Let $\hypclass$ be the set of all threshold functions $h_a$
for $a \in [0, 1]$. Then $\vcdim (\hypclass) = 1$ and  $|\hypclass| = \infty$.

\subsection*{6.7.2} Define $\hypclass$ to consist of the single threshold function
$h_{1/2}$. In this case, $\log_2 |\hypclass| = 0$ and if $a \in [0, 1]$, the 
function $g(a) = 1 - h_{1/2}(a)$ cannot be extended in $\hypclass$. Hence the 
$\vcdim (\hypclass) = 0$, matching the upper bound.
  
  
\section*{Exercise 6.8}

Fix a $d \in \Rone$. We will construct a set $C = \{x_1, 
\ldots, x_d\} \subset [0, 1]$ and show that for any boolean function $(b_1, \ldots, b_d)$ 
on $C$, there exists $\theta \in \Rone$ such that: $\ceiling{\sin (\theta x_i) } = b_i$ 
for all $1 \leq i \leq d$. In particular, we will construct the binary representations
of the elements of $C$. Consider a matrix with $d$ rows and $2^d + 1$ columns, where 
the $i$th row represents the number $x_i$. Fill in the first $2^d$ columns of the matrix
(from top to bottom) with the binary representations of the numbers $0, 1, \ldots, 2^d - 1$.
Finally fill in the last column with ones. The number $x_i = 0.a_1a_2, \ldots, 
a_{2^d}a_{2^d + 1}$, where $a_1, a_2, \ldots, a_{2^d}, a_{2^d + 1}$ are the elements
of the $i$th row of the matrix. This completes the construction of the $d$ numbers from 
$[0, 1]$.

From the way this matrix has been constructed, it is clear that the bit patterns in the 
first $2^d$ columns are all the elements of $\{0, 1\}^d$. Let $(b_1, \ldots, b_d)$ be any 
boolean function defined on the set $C$. Then the bitwise complement $(\bar{b}_1, \ldots, 
\bar{b}_d)$ of the pattern $(b_1, \ldots, b_d)$ is in some column $j$ of the matrix 
created in the last paragraph, where $1 \leq j \leq 2^d$. Define $\theta = 2^j \pi$, and
using the hint provided in the text, we obtain that  for all $1 \leq i \leq d$:
\[
    \ceiling{\sin(2^j \pi x_i)} = 1 - B_j(x_i),  
\]  
where $B_j(x_i)$ is the $j$th bit in the binary representation of $x_i$. Now this bit is 
simply $\bar{b}_i$, since the $j$th column of the matrix is $(\bar{b}_1, \ldots, 
\bar{b}_d)$. Hence the right-hand side of the equation is $1 - \bar{b}_i = b_i$. 
This shows that every boolean function defined on $C$ can be extended in $\hypclass$. Since
one can do this for sets of any size $d$, $\vcdim (\hypclass) = \infty$. 
 

\section*{Exercise 6.9}

The VC-dimension of the class of signed intervals. Let $c_1, c_2, c_3$ be any
three reals with $c_1 < c_2 < c_3$. We wish to show that for any function from
$\{c_1, c_2, c_3\}$ to $\{-1, 1\}$ can be extended to a function in
$\hypclass$.  The only contentious candidates are $(1, -1, 1)$ and $(-1, 1,
-1)$. The first function $(1, -1, 1)$ can be extended to $h_{b_1, b_2, -1}$,
where $b_1$ and $b_2$ are two reals such that $c_1 < b_1 < c_2$ and $c_2 < b_2
< c_3$. Similarly, the second function $(-1, 1, -1)$ can be extended to
$h_{b_1, b_2, +1}$. This shows that $\vcdim (\hypclass) \geq 3$. 

Now consider any set of four reals $c_1, c_2, c_3, c_4$ with $c_1 < c_2 < c_3 <
c_4$.  Then the function $(1, -1, 1, -1)$ cannot be extended to any function in
$\hypclass$.  For such a function to exist, it would have to map an interval
around $c_2$ to $-1$ and the rest of the reals to $1$; but that also map $c_4$
to $1$. This shows that such a function cannot exist and hence 
$\vcdim (\hypclass) \leq 3$. Together with the upper bound, we have 
that $\vcdim(\hypclass) = 3$.  

\section*{Exercise 6.11}

\subsection*{6.11.1} 

Let $\hypclass_1, \ldots, \hypclass_r$ be hypothesis classes over a fixed domain
$\dom$ and let $d := \max_{i} \vcdim (\hypclass_i )$. Suppose that $d \geq 3$. We need 
to show that 
\[
    \vcdim \left (\bigcup_{i = 1}^{r} \hypclass_i \right ) \leq 4d \log (2d) + 2 \log r
\]

Let $C = \{c_1, \ldots, c_k\}$ be a set that is shattered by the union set 
$\bigcup_{i = 1}^r \hypclass_i$. Then all $2^k$ binary functions on $C$ have 
extensions in $\bigcup_{i = 1}^r \hypclass_i$. By Sauer's Lemma, for any 
hypothesis class $\hypclass_i$, $1 \leq i \leq r$, the number of possible 
extensions on a set of size $k$ is $\sum_{i = 0}^d {k \choose i } \leq (e k/d)^d$. 
This is strictly less than $k^d$ since $d \geq 3$. Hence the total number of extensions 
possible in the union on a set of size $k$ is strictly less than $r \cdot k^d$. Hence we must
have $2^k < r \cdot k^d$, which implies that 
$k < d \cdot \log_2 k + \log_2 r$.

Lemma~A.2 states that if $a \geq 1$ and $b > 0$ then $x \geq 4a \log (2a) + 2b$
implies that $x \geq a \log (x) + b$. The contrapositive states that if 
$x < a \log (x) + b$ then $x < 4a \log (2a) + 2b$. If we apply this to our case, 
we obtain that $k < 4d \log (2d) + 2 \log (r)$.

\section*{Exercise 6.12}

\subsection*{6.12.1}

Let $g$ and $\cal F$ be as stated. We first show that 
$\vcdim(\pos(\mathcal{F} + g)) \geq \vcdim (\pos (\mathcal{F}) )$. Let $C \subseteq \Rone^n$ be a 
set that is shattered by $\pos(\mathcal{F})$. Then for every $C' \subseteq C$, there exist
$h_1, h_2 \in \mathcal{F}$ such that for all $x \in C$:
\begin{align*}
    h_1(x) > 0 & \text { iff } x \in C' \\
    h_2(x) > 0 & \text{ iff } x \in C \setminus C'.
\end{align*}
The idea here is to use a linear combination of $h_1$ and $h_2$ and $g$ to obtain a function
that takes on strictly positive values on $C'$ and non-positive values on $C \setminus C'$. 
Define $\alpha_1$ and $\alpha_2$ as follows:
\[
\alpha_1 := 1 + \frac{\max_{x \in C'} |g(x)|}{\min_{x \in C'} h_1(x)}, \qquad 
\alpha_2 := 1 + \frac{{\max_{x \in C \setminus C'} |g(x)|}}{\min_{x \in C \setminus C'} h_2(x)}
\]
Then $\alpha_1 h_1(x) - \alpha_2 h_2(x) + g(x) > 0$ iff $x \in C'$, showing that
$\pos (\mathcal{F} + g)$ also shatters $C$. Hence 
$\vcdim(\pos(\mathcal{F} + g)) \geq \vcdim (\pos (\mathcal{F}) )$.

We next show that $\vcdim(\pos(\mathcal{F} + g)) \leq \vcdim (\pos (\mathcal{F}) )$. Now let 
$C \subseteq \Rone^n$ be shattered by $\pos(\mathcal{F} + g)$. This means that 
for every $C' \subseteq C$, there exist $h_1, h_2 \in \mathcal{F}$ such that for 
all $x \in C$:
\begin{align*}
    h_1(x) + g(x) > 0 & \text { iff } x \in C' \\
    h_2(x) + g(x) > 0 & \text{ iff } x \in C \setminus C'.
\end{align*}
This immediately shows that for all $x \in C$:
\[
    (h_1(x) + g(x)) + (- h_2(x) - g(x)) > 0 \text { iff } x \in C'. 
\]
Hence $\pos (\mathcal{F})$ shatters $C$ and
$\vcdim(\pos(\mathcal{F} + g)) \leq \vcdim (\pos (\mathcal{F}) )$. 

\subsection*{6.12.2}

 We wish to show that $\vcdim (\pos (\mathcal{F})) = \dim (\mathcal{F})$, where 
 $\dim (\mathcal{F})$ is the dimension of $\mathcal{F}$ as a vector space. Let 
 $\mathcal{F}$ be a finite dimensional vector space of dimension~$d$ with basis
 $f_1, \ldots, f_d$. For any $h \in \mathcal{F}$ there exist real numbers 
 $\alpha_1, \ldots, \alpha_d$, not all zero, such that
 \[
    h = \alpha_1 f_1 + \cdots + \alpha_d f_d.
 \]
Note that 
\[
    \pos (\mathcal{F}) = \pos (\{ \alpha_1 f_1 + \cdots + \alpha_d f_d 
        \mid (\alpha_1, \ldots, \alpha_d) \in \Rone^d \setminus (0, 0, \ldots, 0) \}).
\]

