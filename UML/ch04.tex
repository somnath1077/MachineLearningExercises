\chapter{Learning via Uniform Convergence}

\section*{Notes on Chapter 4}

Given any hypothesis class $\hypclass$ and a domain $Z = \dom \times Y$, let
$l$ be a loss function from $\hypclass \times Z \rightarrow \Rpos$. Finally let
$\dist$ be a distribution over the domain $Z$. The risk of a hypothesis $h \in
\hypclass$ is
\[
    L_{\dist}(h) = \Prtwo{z \sim \dist}{l(h, z)}
\]
A training set $S$ is $\epsilon$-representative w.r.t $Z$, $\hypclass$, $Z$ and
$l$ if for all $h \in \hypclass$, $|L_S (h) - L_{\dist} (h)| \leq \epsilon$.
Thus any hypothesis on an $\epsilon$-representative training set has an
in-sample error that is close to their true risk. 

If $S$ is $\epsilon$-representative, then the $\ERM_{\hypclass}(S)$ learning
rule is guaranteed to return a good hypothesis. More specifically,
\begin{lemma}
\label{lemma:epsilon_representative}
Fix a hypothesis class $\hypclass$, a domain $Z = \dom \times Y$, a loss 
function $l \colon \hypclass \times Z \rightarrow \Rpos$ and a distribution
$\dist$ over the domain $Z$. Let $S$ be an $\epsilon/2$-representative sample. 
Then any output $h_S$ of $\ERM_{\hypclass}(S)$ satisfies 
\[
    L_{\dist} (h_S) \leq \min_{h' \in \hypclass} L_{\dist}(h') + \epsilon 
\]
\end{lemma}

Therefore in order for the $\ERM$ rule to be an agnostic PAC-learner, all we
need to do is to ensure that with probability of at least $1 - \delta$ over
random choices of the training set, we end up with an
$\epsilon/2$-representative training sample. This requirement is baked into 
the definition of \emph{uniform convergence}. 

\begin{definition}
A hypothesis class $\hypclass$ is uniformly convergent wrt a domain $Z$ 
and a loss function $l$, if there exists a function 
$\USampleComp \colon (0, 1) \times (0, 1) \rightarrow \Nat$ such that 
for all $\epsilon, \delta \in (0, 1)$ and all distributions $\dist$ on $Z$,
if a sample of at least $\usamplecomp{\epsilon}{\delta}$ examples is chosen
i.i.d from $\dist$, then with probability $1 - \delta$, the sample is 
$\epsilon$-representative.   
\end{definition}

By Lemma~(\ref{lemma:epsilon_representative}), if $\hypclass$ is uniformly convergent
with function $\USampleComp$, then it is agnostically PAC-learnable with sample 
complexity $\samplecomp{\epsilon}{\delta} \leq \usamplecomp{\epsilon / 2}{\delta}$. In 
this case, the $\ERM$ paradigm is a successful agnostic PAC-learner for $\hypclass$.

\section*{Ecercise 4.1}

We first show that $(1) \Rightarrow (2)$. For each $n \in \Nat$, define 
$\epsilon_n = 1 / 2^n$ and $\delta_n = 1 / 2^n$. Then by $(1)$, for each 
$n \in \Nat$, there exists $m(\epsilon_n, \delta_n)$ such that 
$\forall m \geq m(\epsilon_n, \delta_n)$, 
\[
    \Prtwo{S \sim \dist^{m}}{L_{\dist} (h_S) > \epsilon_n} < \delta_n.
\]
